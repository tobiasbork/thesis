
%----------------------------------------------------------------------------------------
%	PACKAGES AND OTHER DOCUMENT CONFIGURATIONS
%----------------------------------------------------------------------------------------

\documentclass[
11pt, % The default document font size, options: 10pt, 11pt, 12pt
%oneside, % Two side (alternating margins) for binding by default, uncomment to switch to one side
ngerman, % ngerman for German
singlespacing, % Single line spacing, alternatives: onehalfspacing or doublespacing
%draft, % Uncomment to enable draft mode (no pictures, no links, overfull hboxes indicated)
%nolistspacing, % If the document is onehalfspacing or doublespacing, uncomment this to set spacing in lists to single
%liststotoc, % Uncomment to add the list of figures/tables/etc to the table of contents
%toctotoc, % Uncomment to add the main table of contents to the table of contents
parskip, % Uncomment to add space between paragraphs
%nohyperref, % Uncomment to not load the hyperref package
headsepline, % Uncomment to get a line under the header
%chapterinoneline, % Uncomment to place the chapter title next to the number on one line
%consistentlayout, % Uncomment to change the layout of the declaration, abstract and acknowledgements pages to match the default layout
]{MastersDoctoralThesis} % The class file specifying the document structure

\usepackage[utf8]{inputenc} % Required for inputting international characters
\usepackage[T1]{fontenc} % Output font encoding for international characters

\usepackage{mathpazo} % Use the Palatino font by default

\usepackage[backend=biber,style=alphabetic,citestyle=alphabetic,natbib=true]{biblatex} % Use the bibtex backend with the authoryear citation style (which resembles APA)

\addbibresource{references.bib} % The filename of the bibliography

\usepackage[autostyle=true]{csquotes} % Required to generate language-dependent quotes in the bibliography

%----------------------------------------------------------------------------------------
%	OWN SETTINGS
%----------------------------------------------------------------------------------------

\usepackage{amsmath}
\usepackage{amsthm}
\usepackage{amsfonts,amssymb}
\usepackage{graphicx}
\usepackage{algorithm}
\usepackage[noend]{algpseudocode}
\usepackage{subcaption}
\usepackage{makecell}
\usepackage{enumerate}
\usepackage{mathtools}
\usepackage{xr}
%\usepackage{bm}

% add external documents for cross-referencing between different files
\externaldocument[introduction_]{chapters/introduction}
\externaldocument[fundamentals_]{chapters/fundamentals}
\externaldocument[pca_]{chapters/pca}
\externaldocument[sparse_pca_]{chapters/sparse_pca}
\externaldocument[implementation_]{chapters/implementation}
\externaldocument[application_]{chapters/application}
\externaldocument[conclusion_]{chapters/conclusion}
\externaldocument[appendix_]{chapters/appendix}

% Declare theorem style and numbering
\theoremstyle{plain}
\newtheorem{thm}{Theorem}[chapter]
\newtheorem{lemma}{Lemma}[chapter]
\newtheorem{corr}{Korrolar}[chapter]
\theoremstyle{definition}
\newtheorem{defn}[thm]{Definition}
\newtheorem{ex}[thm]{Beispiel}

\newcommand{\theadc}[1]{\multicolumn{1}{c}{\textbf{#1}}}
\newcommand{\q}[1]{"#1"}
\newcommand{\mat}[1]{\mathbf{#1}} 
\DeclareMathOperator*{\argmax}{arg\,max}
\DeclareMathOperator*{\argmin}{arg\,min}
\newcommand{\norm}[1]{\left\lVert #1 \right\rVert}
\newcommand{\abs}[1]{\left\lvert #1 \right\rvert}
\newcommand\inner[2]{\langle #1, #2 \rangle}
\newcommand\rang[1]{\text{rank} \left( #1 \right)}
\newcommand\spur[1]{\text{tr} \left( #1 \right)}
\newcommand{\defeq}{\vcentcolon=}
\newcommand{\rnn}{\mathbb{R}^{n \times n}}
\newcommand{\rmn}{\mathbb{R}^{m \times n}}
\newcommand{\rnp}{\mathbb{R}^{n \times p}}
\newcommand{\rn}{\mathbb{R}^n}

\makeatletter
\newenvironment{chapquote}[2][2em]
  {\setlength{\@tempdima}{#1}%
   \def\chapquote@author{#2}%
   \parshape 1 \@tempdima \dimexpr\textwidth-2\@tempdima\relax%
   \itshape}
  {\par\normalfont\hfill--\ \chapquote@author\hspace*{\@tempdima}\par\bigskip}
\makeatother

\captionsetup[subfigure]{width=0.9\textwidth}

\setcounter{secnumdepth}{2}
\setcounter{tocdepth}{2}

%----------------------------------------------------------------------------------------
%	MARGIN SETTINGS
%----------------------------------------------------------------------------------------

\geometry{
	paper=a4paper, % Change to letterpaper for US letter
	inner=2.5cm, % Inner margin
	outer=2.5cm, % Outer margin
	bindingoffset=.5cm, % Binding offset
	top=1.5cm, % Top margin
	bottom=1.5cm, % Bottom margin
	%showframe, % Uncomment to show how the type block is set on the page
}

%----------------------------------------------------------------------------------------
%	THESIS INFORMATION
%----------------------------------------------------------------------------------------


\usepackage{BA_Titelseite}


%Namen des Verfassers der Arbeit
\author{Tobias Bork}
%Geburtsdatum des Verfassers
\geburtsdatum{21. November 1997}
%Gebortsort des Verfassers
\geburtsort{Reutlingen}
%Datum der Abgabe der Arbeit
\date{\today}

%Name des Betreuers
\betreuer{Betreuer: Prof. Dr. Jochen Garcke}

%Name des Zweitgutachters
\zweitgutachter{Zweitgutachter: Prof. Dr. X Y}

%Name des Instituts.
\institut{Institut f\"ur Numerische Simulation}

%Titel der Bachelorarbeit
\title{Analyse dünnbesetzter Hauptachsen für Frequenzdaten}

%Do not change!
\ausarbeitungstyp{Bachelorarbeit Mathematik}

%----------------------------------------------------------------------------------------
%	BEGIN DOCUMENT
%----------------------------------------------------------------------------------------

\begin{document}

\frontmatter % Use roman page numbering style (i, ii, iii, iv...) for the pre-content pages

\pagestyle{plain} % Default to the plain heading style until the thesis style is called for the body content

%----------------------------------------------------------------------------------------
%	TITLE PAGE
%----------------------------------------------------------------------------------------

\maketitle

%----------------------------------------------------------------------------------------
%	ACKNOWLEDGEMENTS
%----------------------------------------------------------------------------------------

\frontmatter
\newgeometry{margin=4cm}
\addtocounter{page}{2}

\begin{acknowledgements}
\addchaptertocentry{\acknowledgementname} % Add the acknowledgements to the table of contents
\vspace{1.5cm}
An dieser Stelle möchte ich mich bei Prof. Dr. Jochen Garcke für die Möglichkeit bedanken, meine Bachelorarbeit mit Anwendung auf einen realen Datensatz schreiben zu dürfen. Des Weiteren danke ich Dr. X Y für die Übernahme der Zweitkorrektur. Dank geht auch an meine beiden Betreuer Dr. Sebastian Mayer und Christian Gscheidle, welche mir bei mathematischen wie physikalischen Fragen zur Seite standen und sich immer für mich Zeit genommen haben.

Für die nötige Motivation im Studium, die gemeinsame Zeit und nicht zuletzt auch für das Korrekturlesen dieser Arbeit bedanke ich mich bei Hendrik Baers, Christopher Reiners und Lennard Schiefelbein.

Zudem bedanke ich mich bei meiner Familie und Maja Peters für die reichlichen Gespräche und Ratschläge während des Studiums. Die bedingungslose Unterstützung hat mir in vielen Situation geholfen, die richtigen Entscheidungen zu treffen. 

\end{acknowledgements}

\restoregeometry

%----------------------------------------------------------------------------------------
%	LIST OF CONTENTS/FIGURES/TABLES PAGES
%----------------------------------------------------------------------------------------

\tableofcontents % Prints the main table of contents

%\listoffigures % Prints the list of figures

%\listoftables % Prints the list of tables


%----------------------------------------------------------------------------------------
%	THESIS CONTENT - CHAPTERS
%----------------------------------------------------------------------------------------

\mainmatter % Begin numeric (1,2,3...) page numbering

\pagestyle{thesis} % Return the page headers back to the "thesis" style

% Main chapter title
\chapter{Einführung}

\cite{elad}
\cite{foucart}
\cite{hastie_elements}
\cite{gribonval}
\cite{jenatton}
\cite{johnstone}
\cite{yata}
\cite{mairal}
\cite{tibshirani_lasso}
\cite{tibshirani_uniqueness}
\cite{zou_elasticnet}
\cite{zou_sparsepca}
\cite{zou_overview}
\cite{efron_lars}

% Chapter label
\label{introduction}

\section{Motivation}

\section{Dimensionsreduktionsverfahren}

\section{Sparse Approximations / Representations}

\section{Interpretierbarkeit}

\section{Compressed Sensing Beispiel}
% Main chapter title
\chapter{Mathematische Grundlagen}

% Chapter label
\label{fundamentals}

\section{Normen und deren Effekte}

\subsection{l0-Norm}
\subsection{l1-Norm}
\subsection{l2-Norm}

\section{Regression}
Lineare Regression (Least Squares)
\subsection{LASSO}
\subsection{Ridge Regression}

\section{Lineare Algebra}

\subsection{Orthogonalprojektion}
\begin{defn}
Zwei Vektoren $\vec a$ und $\vec b$ sind genau dann orthogonal, wenn ihr Skalarprodukt null ist, also
$$\vec a \perp \vec b \iff\vec a \cdot \vec b = 0.$$
\end{defn}

Was sind orthogonale, orthonormale Matrizen, orthogonale, orthonormale Basis?
Skalarprodukt?
Von einem Skalarprodukt induzierte Norm?
Projektionsmatrizen?

Allgemeine orthogonale Projektionsmatrix falls keine ONB gegeben ist.
$$\mat P_{\mat A} = \mat A (\mat A^T \mat A)^{-1} \mat A$$

Von Wikipedia:
\begin{defn}
Eine Orthogonalprojektion auf einen Untervektorraum $U$ eines Vektorraums $V$ ist eine lineare Abbildung $P_U \colon V \rightarrow V$, die für alle Vektoren $v\in V$ die beiden Eigenschaften

\begin{itemize}
\item $P_U(v) \in U \quad$   (Projektion)
\item $\langle P_U(v) - v , u \rangle = 0$ für alle $u \in U \quad$ (Orthogonalität)
\end{itemize}
erfüllt.
\end{defn}

Allgemeine orthogonale Projektion auf einen affinen linearen Unterraum.
$$P_{U_0}(v) = r_0 + \sum_{i=1}^k \frac{\langle v - r_0, w_i \rangle}{\langle w_i, w_i \rangle} w_i$$

WÖRTLICH VON WIKIPEDIA:
Der orthogonal projizierte Vektor minimiert den Abstand zwischen dem Ausgangsvektor und allen Vektoren des Untervektorraums bezüglich der von dem Skalarprodukt abgeleiteten Norm $\norm{\cdot}$, denn es gilt mit dem Satz des Pythagoras für Skalarprodukträume

$$\norm{u - v}^2 = \norm{u - P_U(v)}^2 + \norm{P_U(v) - v}^2 \geq \norm{P_U(v) - v}^2$$

\subsection{Matrixzerlegungen}

Diagonalisierbarkeit?

\subsection{Eigenwertzerlegung}
\subsubsection{Eigenwerte, Eigenvektoren}
\subsection{Singulärwertzerlegung}
\subsubsection{Singulärwerte}

\subsection{Matrixnorm}
\begin{defn}[Frobeniusnorm]
$$\norm{\mat X}_F$$
\end{defn}

\subsection{Rang}

\begin{defn}[Rang]
Eine Matrix hat Rang k ... wenn
\end{defn}

\begin{thm}[Wörtlich von Wikipedia, Eckart-Young-Theorem]
The unstructured problem with fit measured by the Frobenius norm, i.e.,
$$\text{minimize} \quad \text{over } \widehat D \quad \|D - \widehat D\|_{\text{F}} \quad\text{subject to}\quad \operatorname{rank}\big(\widehat D\big) \leq r $$
has analytic solution in terms of the singular value decomposition of the data matrix. The result is referred to as the matrix approximation lemma or Eckart–Young–Mirsky theorem.[4] Let

$$D = U\Sigma V^{\top} \in \mathbb{R}^{m\times n}, \quad m \leq n$$

be the singular value decomposition of $D$ and partition $U, \Sigma=:\operatorname{diag}(\sigma_1,\ldots,\sigma_m)$, and $V$ as follows:

$$U =: \begin{bmatrix} U_1 & U_2\end{bmatrix}, \quad 
\Sigma =: \begin{bmatrix} \Sigma_1 & 0 \\ 0 & \Sigma_2 \end{bmatrix}, \quad\text{and}\quad 
V =: \begin{bmatrix} V_1 & V_2 \end{bmatrix},$$

where $U_{1}$ is $m\times r$, $\Sigma _{1}$ is $r\times r$, and $V_{1}$ is $n\times r$. Then the rank-$r$ matrix, obtained from the truncated singular value decomposition

$$\widehat D^* = U_1 \Sigma_1 V_1^{\top},$$

is such that

$$\|D-\widehat D^*\|_{\text{F}} = \min_{\operatorname{rank}(\widehat D) \leq r} \|D-\widehat D\|_{\text{F}} = \sqrt{\sigma^2_{r+1} + \cdots + \sigma^2_m}.$$

The minimizer $\widehat D^*$ is unique if and only if $\sigma_{r+1}\neq\sigma_{r}$.
\end{thm}

\section{Signaltheorie}

\subsection{Fouriertransformation}
\subsection{Nyquist-Shannon Abtasttheorem}

\section{Statistik}
Varianz, Erwartungswert
\subsection{Empirische Kovarianzmatrix}

\section{Mannigfaltigkeit}

\section{Dictionary Learning}
% Main chapter title
\chapter{Hauptkomponentenanalyse}

% Chapter label
\label{pca}

\section{Motivation}

Die Hauptkomponentenanalyse ist ein weitverbreites multivariates statistisches Verfahren zur Dimensionsreduktion. Multivariate Verfahren zielen darauf ab, die in einem Datensatz enthaltene Zahl der Variablen zu verringern, ohne die darin enthaltene Information wesentlich zu reduzieren. Dadurch können umfangreiche Datensätze strukturiert, veranschaulicht und vereinfacht werden. Als Teil der explorativen Statistik ...

Somit findet die Hauptkomponentenanalyse in vielen Bereichen Anwendung.
Ein paar Beispiele (hand written zip code classification or human face recognition).

Das dahinterstehende mathematische Problem kann auf mehrere Weisen beschrieben werden. Zunächst wollen wir die Hauptkomponentenanalyse so konstruieren, dass die Idee des minimalen Informationsverlust im Vordergrund steht. Anschließend werden wir das Problem auf ein Singulärwertzerlegung zurückführen, die auch zur effizienten Implementierung genutzt wird. Des Weiteren werden wir die Hauptkomponentenanalyse als Regressionsproblem betrachten, und die geometrische Interpretation weiter verdeutlichen. Zu Schluss werden wir die Äquivalenz dieser Formulierungen und einige theoretische Aussagen zeigen.

\section{Einführung}

Was heißt überhaupt Hauptkomponente und was ist eine Hauptachse vielleicht an dieser Stelle.

Die zentrale Idee der Hauptkomponentenanalyse besteht darin, die bestehenden Variablen in neue, unkorrelierte Variablen zu überführen, ohne dabei Information zu verlieren. Als Maß für den Informationsgehalt der Daten wird hierbei die Varianz verwendet. Konkret konstruieren wir Variablen, die sich aus Linerakombinationen der Alten zusammensetzen. Dabei sollen die neuen Variablen der Wichtigkeit nach sortiert sein. In anderen Worten enthält die erste Variable die meiste Information bzw. die größte Varianz, dann die zweite, usw.
Die eigentliche Dimensionsreduktion findet dann durch Selektierung statt. Je nach Komplexität des Modells und Informationsverlust können so mehr oder weniger ausgewählt werden. Somit haben wir eine kleine, neue Menge an Variablen, die aber trotzdem den Großteil an Informationen / Varianz beinhaltet.




Wie müssen die Daten aufbereitet sein? Zentriert und skaliert? Erklären verschiedener Methoden und deren Auswirkungen für die Daten. Korrelationsmatrix oder Kovarianzmatrix?




\section{Konstruktion}

Gegeben sei eine Matrix $\mat{X} \in \mathbb{R}^{n\times p}$, wobei $n$ die Anzahl der Samples bzw. Beobachtungen und $p$ die Anzahl der Variablen. Ohne Beschränkung der Allgemeinheit nehmen wir im Folgenden an, dass die Spaltendurchschnitte der Matrix 0 sind, d.h. jede Variable einzeln zentriert ist. Falls dies nicht der Fall ist können wir die Variablen einfach zentrieren. (Eventuell erwähnen, was passiert wenn man die Variablen nicht zentriert)
Aufgabe der Hauptkomponentenanalyse ist es nun

\subsection{Problemformulierung als Varianzmaximierung}
Die erste Hauptachse wird definiert als $$v_1 = \max_{\norm{v}_2 = 1} v^T \sum v$$. Die Hauptkomponente wird dann definiert durch $Z_1 = \sum_{j=1}^{p} \alpha_{1j}X_j$, wobei $\alpha_1 = (\alpha_{11}, \ldots, \alpha_{1p})^T$
wobei $\sum = \dfrac{(\mat{X}^T\mat{X})}{n}$ die Kovarianzmatrix ist. Anschließend werden die restlichen Hauptachsen sequentiell definiert.
$$\alpha_{k+1} = \argmax_{\norm{\alpha} = 1} \alpha^T\sum\alpha$$ unter der Bedingung, dass $\alpha^T\alpha_l = 0, \forall 1 \leq l \leq k$.

\cite{zou_overview}

Fasst man diese Schritte zusammen kann man eine Eigenwertzerlegung vornehmen. Theorem, dass die Eigenwerte von $X^TX$ die Varianz maximieren.

\subsection{Formulierung als Singulärwertzerlegung}
$$ \mat{X} = \mat{U}\mat{V}\mat{D}^T $$
wobei $\mat{D}$ eine Diagonalmatrix mit Elementen $d_1,\ldots,d_p$ in absteigender Reihenfolge, $\mat{U}$ eine $n \times p$ und $\mat{V}$ eine $p \times p$ orthogonale Matrix.
$\mat{U}\mat{V}$ sind die Hauptkomponenten und die Spalten von $\mat{V}$ sind die Eigenvektoren von $\mat{X}$.

\subsection{Formulierung als Regressionsproblem}
$$\hat{\mat{A}} = \argmin_{\mat{A}} \sum_{i=1}^{n} \norm{x_i - \mat{A}\mat{A}^Tx_i}^2 + \lambda \sum_{j=1}^{k}\norm{\beta_j}^2$$

subject to $\mat{A}^T\mat{A} = I_{k \times k}$

\cite{zou_sparsepca}

Man projiziert die Daten auf einen k-dimensionalen linearen Unterraum. Man kann zeigen, dass die Lösung dieses Problem genau die ersten k Hauptachsen sind.

Ausgehend von diesem Regressionsproblem werden wir im nächsten Kapitel die Variante der dünnbesetzten Hauptkomponentenanalyse beschreiben.

\section{Theoretische Aussagen}

\begin{thm}
PCA always gives unique solution.
\end{thm}

\begin{thm}
PCA inconsistent for p >> n.
\end{thm}

\begin{defn}
This is a definition
\end{defn}
% Main chapter title
\chapter{Dünnbesetzte Hauptkomponentenanalyse}

% Chapter label
\label{sparse_pca}

Ein wesentlicher Nachteil der Hauptkomponentenanalyse besteht darin, dass sich die neuen Variablen aus einer Linearkombination \textit{aller} bestehenden Variablen zusammensetzt. Dies erschwert besonders für hochdimensionale Daten eine Interpretation der Hauptachsen. Während zuvor jede Variable eine Bedeutung hatte, sind wir nach der Transformation meist nicht in der Lage den Hauptachsen eine Bedeutung im Kontext zuzuweisen. Um zu verstehen, was die Hauptachsen im Modell repräsentieren kann es besonders hilfreich sein, wenn diese \textit{dünnbesetzt} sind, sich also nur aus wenigen Variablen zusammensetzen. Treffen wir irgendwelche Annahmen? Des Weiteren ist nicht jede Variable relevant zur Strukturerkennung. impose extra constraints, which sacrifices some variance in order to improve interpretability. Interpretation ist oberstes Ziel!!!

shape/image analysis, ecological studys und neuroscience data application.

Zu Anfang dieses Kapitels werden wir eine naheliegende mathematische Formulierung des Problems beschreiben. Leider wird sich diese als NP-vollständig herausstellen, weshalb wir in Abschnitt \ref{relaxation} verschiedene Wege aufzeigen, dass Problem zu relaxieren. In \ref{sparse_pca_construction} möchten wir uns mit einem dieser Ansätze intensiv beschäftigen, welcher den Ausgangspunkt für den weiteren Verlauf dieser Arbeit darstellt. Wir haben uns dazu entschieden, den von Zou, Hastie und Tibshirani in \cite{zou_sparsepca} eingeführten Ansatz für diesen Zweck zu verfolgen. Dieser gilt sicherlich zu der am meisten verbreiteten Variante der dünnbesetzten Hauptkomponentenanalyse. Kürzlich erschienenes Paper, Neuerungen. Der Rest dieses Kapitels ist den Details dieses Ansatzes gewidmet.


%----------------------------------------------------------------------------------------
%	Problemformulierung
%----------------------------------------------------------------------------------------


\section{Problemformulierung}
\label{problem_formulation}

Wir möchten nun Hauptachsen eines gegebenen Datensatzes identifizieren mit der Zusatzbedingung, dass diese dünnbesetzt sind. Die wohl einfachste Vorgehensweise ist, zuerst die gewöhnliche Hauptkomponentenanalyse durchzuführen und anschließend ein Schwellwertmethode auf die Hauptachsen anzuwenden. Hierbei vernachlässigt man alle Koeffizienten, die kleiner als ein bestimmter Schwellenwert sind, indem man sie auf 0 setzt. Eine solche Prozedur kann aber in vielen Fällen irreführend sein, unter welcher die Qualität der Ergebnisse leidet \cite{cadima}. Die Wichtigkeit einer Variable in den Hauptachsen wird nicht allein durch den Koeffizient bestimmt. Zu berücksichtigen sind unter anderem sowohl die Standardabweichung als auch die Korrelationen mit anderen Variablen. Bei einer Schwellwertmethode werden diese Faktoren nicht beachtet, weshalb den Ergebnissen im Allgemeinen nicht vertraut werden darf.

Hier Regression on ordinary PCA's mit Zou et al?

Anstelle eines zweischrittigen Ansatzes kann die Dünnbesetzung direkt in die Problemformulierung mit eingebaut werden. Gegeben sei dazu wieder eine Datenmatrix $\mat X \in \rnp$, wobei $n$ die Anzahl an Beobachtungen und $p$ die Anzahl an Variablen ist. Des Weiteren gehen wir davon aus, dass die Matrix $\mat X$ zuvor spaltenweise zentriert wurde. Dann kann die dünnbesetzte Hauptkomponentenanalyse als sukzessives Maximierungsproblem formuliert werden:
\begin{gather}
\label{sparse_pca_np}
\begin{split}
v_{k} = \argmax_{\norm{v}_2 = 1} v^{T}\mat{\Sigma} v\\
\text{unter der Nebendingung, dass für } k\geq 2 \, v_{k}^Tv_{l} = 0 \quad \forall 1 \leq l < k\\
\text{und } \norm{v_{k}}_0 \leq t 
\end{split}
\end{gather}
wobei $\mat{\Sigma} = \frac{\mat X^T \mat X}{n-1}$ die Stichprobenkovarianzmatrix ist. Der einzige Unterschied zur klassischen Hauptkomponentenanalyse, wie wir sie in (\ref{pca_variance_maximization_first}) beschrieben haben, besteht in der Einführung der $\ell_0$-Norm. Somit beschränken wir uns auf die Suche von Hauptachsen, welche höchstens $t$ von Null verschiedene Einträge haben. Wählen wir $t = p$ reduziert sich das Problem auf (\ref{pca_variance_maximization_first}). Während (\ref{sparse_pca_np}) eine sehr schöne und einfache mathematische Formulierung ist, wurde gezeigt, dass dieses Problem NP-vollständig ist \cite{foucart}. Zur Berechnung dünnbesetzter Hauptachsen sind wir also angehalten eine geeignete Relaxation zu finden.


%----------------------------------------------------------------------------------------
%	Relaxation
%----------------------------------------------------------------------------------------


\section{Relaxation}
\label{relaxation}

Es existiert eine Vielfalt an Ansätzen, um das Problem zu relaxieren. Wir wollen zunächst einen kleinen Überblick über die unterschiedlichen Ideen geben und uns anschließend mit einer genauer beschäftigen. Eine selektive Übersicht der verschiedenen Ansätze haben wir hier erstellt.\\

\textbf{SCoTLASS}

Inspiriert von der Lasso Regression \cite{tibshirani_lasso} schlugen Jolliffe et al. \cite{scotlass} vor, die $\ell_1$-Norm anstelle der $\ell_0$-Norm als Strafterm zu verwenden. Wie wir bereits in Abschnitt \ref{lasso} gesehen haben, kann die $\ell_1$-Norm genutzt werden, um dünnbesetzte Vektoren zu erhalten. Somit liegt es nahe das Problem wie folgt zu formulieren.
\begin{gather}
\label{scotlass}
\begin{split}
v_{k} = \argmax_{\norm{v}_2 = 1} v^{T}\mat{\Sigma} v\\
\text{unter der Nebendingung, dass für } k\geq 2 \, v_{k}^Tv_{l} = 0 \quad \forall 1 \leq l < k\\
\text{und } \norm{v_{k}}_1 \leq t 
\end{split}
\end{gather}
Wie in (\ref{sparse_pca_np}) hat man mit der Wahl der Parameters $t$ Einfluss auf die Dünnbesetzung der Hauptachsen. Aufgrund der hohen Berechnungskosten ist SCoTLASS allerdings für hochdimensionale Daten ungeeignet. Diese sind vor allem darauf zurückzuführen, dass (\ref{scotlass}) kein konvexes Optimierungsproblem ist. Des Weiteren ergeben sich Schwierigkeiten bei der Wahl des Hyperparameters $t$. Auch wenn eine passende Wahl eine gewünschte Dünnbesetzung hervorruft, gibt es kaum Orientierungshilfen. Zusätzlich hat ScoTLASS dasselbe grundlegende Problem wie das Lasso. Die Anzahl von null verschiedener Einträge ist durch die Anzahl Beobachtungen im Datensatz limitiert, welches die Brauchbarkeit des Modells deutlich einschränkt. Zusammen mit den hohen Berechnungskosten ist dieser Ansatz in der Praxis daher meist impraktikabel.\\

\textbf{Semidefinite Programmierung}

Konvexe Relaxation ist eine Standard-Technik, um mit schwierigen nichtkonvexen Problemen umzugehen. d'Aspremont et al. \cite{daspremont_semidefinite} entwickeln einen Ansatz, welcher sich als semidefinites Programmierungsproblem ausdrücken lässt. Zunächst werden wir (\ref{sparse_pca_np}) dafür mit Matrizen reformulieren.

Sei $\mat V = v_kv_k^{\top}$. Dann übersetzen sich die Nebenbedingungen 
\begin{align}
\label{semidefinite_programming_naive}
\begin{split}
\argmax_{\mat P} = \spur{\mat \Sigma\mat P}\\
\spur{\mat P} = 1, \quad \norm{\mat P}_0 \leq k^2, \quad \mat P \geq 0, \quad \rang{\mat P} = 1
\end{split}
\end{align}

Diese Formulierung ist noch immer nichtkonvex aufgrund der Rang-Bedingung  und der $\ell_0$-Strafterm. 
Per Definition ist $\mat P$ symmetrisch und $\mat P^2 = \mat P$. Somit ist
$$\norm{\mat P}_{F}^{2} = \spur{\mat P^{\top}\mat P} = \spur{\mat P} = 1$$
und mit der Cauchy-Schwarz-Ungleichung folgt
$$\mat{1}_p^{\top} |\mat P| \mat{1}_p \leq \sqrt{\norm{\mat{P}}_0 \norm{\mat{P}}_F^2} \leq k$$
Ersetzen wir die $\ell_0$-Strafterm und lassen die Rang-Bedingung fallen erhalten wir die DSPCA-Formulierung
\begin{align}
\label{semidefinite_programming}
\begin{split}
\argmax_{\mat P} = \spur{\mat \Sigma\mat P}\\
\spur{\mat P} = 1, \quad \mat{1}_p^{\top} |\mat P| \mat{1}_p \leq k, \quad \mat P \geq 0
\end{split}
\end{align}
Dies stellt ein semidefinites Programmierungsproblem dar, bei welcher die zu optimierenden Variablen symmetrische Matrizen sind unter der Nebenbedingung, dass sie positiv semidefinit sind. Für kleine Probleme kann (\ref{semidefinite_programming}) effizient durch \textit{Innere-Punkte-Verfahren} (English: \textit{interior-point methods}) gelöst werden. SDPT3 \cite{toh}.

In (\ref{semidefinite_programming}) wird allerdings $\mat P$ berechnet und nicht die eigentliche Hauptachse. Hierfür kürzen d'Aspremont et al. die Matrix $\mat P$ und behalten nur den größten Eigenvektor $v_k$. Anschließend erhält man weitere Hauptachsen durch Matrix Deflation, indem wir $\mat \Sigma$ durch
$$\mat \Sigma - (v_k^{\top}\mat \Sigma v_k) v_kv_k^{\top}$$ 
ersetzen. Für größere Probleme wird eine Methode von Nesterov benutzt, um eine Laufzeit von $\mathcal{O}(\frac{p^4\sqrt{\log{p}}}{\epsilon})$ zu erreichen.\\

\textbf{Iterative Schwellenwert-Methode}

Basierend auf der Formulierung (\ref{pca_best_rank_approximation}) der Hauptkomponentenanalyse als beste Rang $k$ Approximation an die Datenmatrix $\mat X$ haben Shen und Huang \cite{shen} das folgende Optimierungsproblem formuliert
\begin{align}
\label{iterative_thresholding}
\begin{split}
(u_1, v_1) = \argmin_{u, v} \norm{\mat X - u v^{\top}}_{F}^{2}  + \lambda \norm{v}_{1}\\
\norm{u}_2 = 1
\end{split}
\end{align}
Somit erhält man mit $\frac{v_1}{\norm{v_1}}$ die erste dünnbesetzte Hauptachse. Auch hier werden die restlichen Hauptachsen sequentiell berechnet durch Ersetzen der Datenmatrix $\mat X_{(k+1)} = \mat X - \sum_{i=1}^k u_iv_i^{\top}$. Jede Iteration kann durch ein alternierendes Minimierungsverfahren gelöst werden. Fixiert man $v$, so ist das optimale $u$ gegeben durch $u = \frac{\mat Xv}{\norm{\mat Xv}}$. Andererseits reduziert sich (\ref{iterative_thresholding}) für festes $u$ auf
$$\argmin_{v} -2\spur{\mat X^{\top}uv^{\top}} + \norm{v}^{2} + \lambda \norm{v}_1.$$
Eine explizite Lösung ist durch den soft-thresholding Operator gegeben
$$v = \operatorname{soft}_{\frac{\lambda}{2}}(\mat X^{\top} U)$$
welcher in Abschnitt \ref{generalized_linear_models} eingeführt worden ist.

Diese Methode ist sehr ähnlich zu der von Zou et al. \cite{zou_sparsepca} , mit welcher wir uns im folgenden Abschnitt beschäftigen werden. Der große Unterschied besteht darin, dass die Hauptachsen dort nicht sequentiell, sondern gleichzeitig berechnet werden. Witten et al. haben in \cite{witten} ebenfalls eine Methode entwickelt, die unter diese Kategorie fällt.\\

\textbf{Weitere Relaxationsideen}

Es gibt noch eine Reihe weiterer Ideen, die in der Literatur betrachtet wurden. Ein interessierter Lesen sei auf die folgenden Ansätze verwiesen.
\begin{itemize}
\item eine verallgemeinerte Potenzmethode \cite{journee}
\item ein alternierendes Maximierungs-Netzwerk \cite{richtarik}
\item Vorwärts und Rückwärts-Greedy-Suche mittels Branch-and-Bound-Verfahren \cite{moghaddam}
\item eine Bayes-Formulierung \cite{guan}
\end{itemize}


%----------------------------------------------------------------------------------------
%	Konstruktion Sparse PCA
%----------------------------------------------------------------------------------------


\section{Konstruktion}
\label{construction}

Wir werden uns nun mit dem von Zou, Hastie und Tibshirani in \cite{zou_sparsepca} eingeführten Ansatz ausführlich beschäftigen. Zou und Hastie führten zuvor in \cite{zou_elasticnet} das sog. \textit{elastic net} ein, welches den Grundstein für die mathematische Formulierung legen wird.

Wie bereits in (\ref{pca_regression_formulation}) beschrieben kann die Hauptkomponentenanalyse als regressionsartiges Problem betrachtet werden. Das folgende Theorem erweitert die bisherige Formulierung, indem nun nicht ausschließlich orthogonale Projektionen erlaubt werden. 
Im Folgenden bezeichnet $k$ die Anzahl an Hauptkomponenten, die wir extrahieren möchten und $x_i$ die $i$-te Zeile von $\mat X$. Wir bezeichnen mit $\mat B$ im Folgenden die dünnbesetzten Hauptachsen, um sie von den klassischen Hauptachsen $\mat V$ zu unterscheiden.

\begin{thm} \label{pca_regression_formulation_ridge}
Sei $\mat{A}_{p \times k} = [ \alpha_1, \ldots ,\alpha_k ]$ und $\mat{B}_{p \times k} = [ \beta_1, \ldots ,\beta_k ]$. Für ein $\lambda_2 > 0$ sei
$$(\hat{\mat{A}}, \hat{\mat{B}}) = \argmin_{\mat{A}, \mat{B}} \sum_{i=1}^{n} \norm{x_i - \mat{A}\mat{B}^Tx_i}^2 + \lambda_2 \sum_{j=1}^{k}\norm{\beta_j}^2$$
$$\text{wobei } \mat{A}^T\mat{A} = I_{k \times k}.$$
Dann ist $\hat{\beta}_j \propto v_j$ für $j = 1,2,\ldots,k$. 
\end{thm}

Fordern wir $\mat A =  \mat B$ reduziert sich die Verlustfunktion $\sum_{i=1}^{n} \norm{x_i - \mat{A}\mat{B}^Tx_i}^2$ auf die klassische Hauptkomponentenanalyse (\ref{pca_regression_formulation}). Theorem \ref{pca_regression_formulation_ridge} zeigt, dass wir die Bedingung $\mat A = \mat B$ unter Einführung eines Ridge-Strafterms vernachlässigen können. Mithilfe dieser Verallgemeinerung können wir die Hauptkomponentenanalyse flexibel modifizieren.

Um dünnbesetzte Hauptachsen zu erhalten können wir einen $\ell_1$-Strafterm in die Zielfunktion einbetten. Das ein solcher Strafterm eine gewünschte Dünnbesetzung hervorruft, haben wir in Abschnitt \ref{lasso} beobachten können. Wir definieren daher das \textit{Sparse PCA Kriterium} mit den Hyperparameter $\lambda_{1,j}$ und $\lambda_2$
\begin{align}
\label{spca_criterion}
\begin{split}
(\hat{\mat{A}}, \hat{\mat{B}}) = \argmin_{\mat{A}, \mat{B}} \sum_{i=1}^{n} \norm{x_i - \mat{A}\mat{B}^Tx_i}_2^2 + \lambda_2 \sum_{j=1}^{k}\norm{\beta_j}_2^2 + \sum_{j=1}^k \lambda_{1,j} \norm{\beta_j}_1\\
\text{unter der Nebenbedingung, dass } \mat{A}^T\mat{A} = I_{k \times k}
\end{split}
\end{align}
Die normierten Spalten von $\mat B$ nennen wir dann die \textit{dünnbesetzten Hauptachsen}. Um die Dünnbesetzung für jede Hauptachse unterschiedlich wählen zu können, erlauben wir unterschiedliche Bestrafungen $\lambda_{1,j}$. Dagegen erlauben wir für die Ridge-Bestrafung $\lambda_2$, die im Wesentlichen für die Reduktion von (\ref{spca_criterion}) auf (\ref{pca_regression_formulation}) benötigt wird falls $\lambda_{1,j} = 0$, keine differenzierte Behandlung. Allerdings hat die $\ell_2$-Bestrafung noch einen weiteren Vorteil, welcher in der Praxis relevant ist. Es bewältigt das Lasso-Defizit, so dass auch mehr als $n$ Variablen im Fall $p>n$ ausgewählt werden können.

Wir möchten darauf hinweisen, dass im Gegensatz zu manch anderen Varianten der dünnbesetzten Hauptkomponentenanalyse (\ref{spca_criterion}) eine zeitgleiche anstatt einer sequentiellen Berechnung der Hauptachsen ermöglicht. Dies wird im folgendem Abschnitt von entscheidender Bedeutung sein.



%----------------------------------------------------------------------------------------
%	Anpassung der Transformation, Residuen und Varianzen
%----------------------------------------------------------------------------------------


\section{Anpassung der Transformation, Residuen und Varianzen}

Bei der Verwendung der dünnbesetzten Hauptkomponentenanalyse übertragen sich viele der Eigenschaften der klassischen Variante nicht. \cite{camacho}. Daher gilt es folgende Punkte zu berücksichtigen.\\

\textbf{Korrelation der Hauptkomponenten}

Bei einer klassischen Hauptkomponentenanalyse sind die Hauptkomponenten aufgrund der orthogonalen Hauptachsen unkorreliert. Letztere Eigenschaft fordern wir bei der dünnbesetzten Variante in (\ref{spca_criterion}) nicht, so dass durchaus starke Korrelationen zwischen den Hauptkomponenten auftreten können. Während dies eine flexiblere Modellierung ermöglicht, wird es dadurch schwieriger die Ergebnisse geeignet zu visualisieren. Besonders bei der Verwendung von Streudiagrammen, welche genutzt werden, um den Beitrag der Ausgangsvariablen zu den Hauptachsen zu visualisieren, kann dies zu Problemen führen. Hierbei unterstellt man die Orthogonalität der Hauptachsen, was zu Verzerrungen der Distanzen im Bild führen kann \cite{geladi}. Des Weiteren kann die Berechnung der erfassten Varianz des Datensatzes, welches häufig als Maß für die Qualität eines Modells genutzt wird, nicht analog zur klassischen Variante durchgeführt werden.\\

\textbf{Varianzverlust der Hauptkomponenten}

Der Erfolg der Hauptkomponentenanalyse beruht vor allem darauf, dass die Hauptkomponenten optimal bezüglich erklärter Varianz sind. Oft kann ein Großteil an Information eines Datensatzes durch eine geringe Anzahl an Hauptkomponenten beschrieben werden, welches eine Visualisierung und Interpretation hochdimensionaler Daten ermöglicht. Bei der dünnbesetzten Hauptkomponentenanalyse opfern wir einen Teil der erklärten Varianz für simplere, einfacher zu interpretierende Hauptachsen. Um einen genauso großen Teil an Information des Datensatzes zu erklären, benötigen wir daher eine größere Anzahl an Hauptkomponenten in unserem Modell. Somit können wir die Dimension des Datensatzes unter Umständen nicht all zu stark reduzieren.

In einem erst vor Kurzem erschienen wissenschaftlichen Artikel zeigen Camacho et al. \cite{camacho}, dass viele der Varianten der dünnbesetzten Hauptkomponentenanalyse bezüglich der genannten Punkte Probleme aufweisen. Insbesondere wurde die Berechnung der Hauptkomponenten, Residuen und der erklärten Varianz bislang falsch durchgeführt. Wir möchten an dieser Stelle die Unterschiede detailliert erklären.

Typischerweise wurden bislang die Hauptkomponenten $\mat Z$ wie bei der klassischen Hauptkomponentenanalyse berechnet, indem man $\mat Z = \mat X \mat B$ setzt, wobei $\mat B$ die Matrix der dünnbesetzten Hauptachsen ist \cite{zou_sparsepca}. Allerdings vernachlässigt man in diesem Fall, dass die Hauptachsen nicht orthogonal zueinander sind. Dies wird deutlich, wenn wir die Hauptkomponentenanalyse wie in (\ref{pca_minimize_reconstruction_error}) als eine bestmögliche Rekonstruktion der Datenmatrix $\mat X$ auffassen.
\begin{align}
\label{pca_scores_loadings_model}
\mat X = \mat Z \mat B^T + \mat E,
\end{align}
wobei $\mat E$ die Matrix der Residuen ist, welche uns erst nach der Berechnung von $\mat Z$ und $\mat B$ zur Verfügung stehen. Im Folgenden sei durch $\mat X = \mat Z \mat B^T$ eine volle Rang Approximation gegeben. Bei der klassischen Hauptkomponentenanalyse multipliziert man beide Seiten mit $\mat B$, um die Hauptkomponenten $\mat X \mat B = \mat Z \mat B^T \mat B = \mat Z$ zu erhalten. Falls die Hauptachsen nicht orthogonal zueinander sind, ist der letzte Schritt nicht mehr gültig. Daher korrigiert man bei der dünnbesetzten Variante mit der Moore-Penrose-Inversen $(\mat B^T \mat B)^+$ ähnlich zu der Methode der kleinsten Quadrate. Demnach sollten die Hauptkomponenten durch
\begin{align}
\label{corrected_scores_sparse_pca}
\mat Z = \mat X \mat B^T (\mat B^T \mat B)^+
\end{align}
berechnet werden. Falls $\mat B$ orthogonal ist, reduziert sich (\ref{corrected_scores_sparse_pca}) wie gehabt auf $\mat Z = \mat X \mat B$. Ein weiterer Unterschied ist, dass nun keine sequentielle Berechnung der Hauptkomponenten mehr möglich ist. Jede Hauptkomponente hängt durch (\ref{corrected_scores_sparse_pca}) von allen Hauptachsen ab, weshalb sich die Werte bei Hinzunahme weiterer Hauptachsen zum Modell jedes mal ändern können. Camacho et al. zeigen empirisch, dass die Korrelation zwischen den Hauptkomponenten für der von uns betrachteten Variante der dünnbesetzten Hauptkomponentenanalyse deutlich sinkt, was eine Interpretation weiter verbessert.

Für die Modellbewertung wird oft die erklärte Varianz des Datensatzes herangezogen, wie in Abschnitt \ref{selection_principal_components} beschrieben. Zou et al. erkennen, dass aufgrund der Korrelation der Hauptkomponenten die Varianzen nicht wie gewohnt errechnet werden können und schlagen in \cite{zou_sparsepca} folgende Methode vor. Die erklärte Varianz für die ersten $j+1$ Hauptkomponenten sollte sich aus der Summe der ersten $j$ zusammen mit der erklärten Varianz der $k+1$-ten Hauptkomponente $Z_{j+1}$ ergeben. Aufgrund der Korrelation der Hauptkomponenten erhält die Varianz von $Z_{j+1}$ aber Beiträge anderer Hauptkomponenten. Um nur die zusätzlich durch $Z_{j+1}$ erhaltene Varianz zu erhalten und lineare Abhängigkeiten zu entfernen, nutzen Zou et al. eine Projektion
\begin{align}
\label{zou_orthogonal_projection}
Z_{j\cdot 1, \ldots, j-1} = Z_j - \mat P_{1, \ldots, j-1}Z_j
\end{align}
wobei $\mat P_{1, \ldots, j-1}$ die orthogonale Projektionsmatrix auf $\{Z_i\}_1^{j-1}$ sei. Mit $Z_{j\cdot 1, \ldots, j-1}$ bezeichnen wir also die Residuen nach Anpassung von $Z_j$ durch $Z_1, \ldots, Z_{j-1}$. Man beachte, dass (\ref{zou_orthogonal_projection}) von der Reihenfolge der $Z_i$ abhängt. Aufgrund der natürlichen Ordnung bei der Hauptkomponentenanalyse stellt dies aber kein Problem dar.
Somit ergibt sich die Gesamtvarianz der ersten $k$ Hauptkomponenten durch 
\begin{align}
\label{zou_explained_variance}
\sum_{j=1}^k \norm{Z_{j\cdot 1, \ldots, j-1}}^2.
\end{align}
Mithilfe einer QR-Zerlegung von $\mat Z = \mat Q \mat R$, wobei $\mat Q$ orthonormal und $\mat R$ eine recht obere Dreiecksmatrix ist, können wir (\ref{zou_explained_variance}) schnell berechnen, denn $\norm{Z_{j\cdot 1, \ldots, j-1}}^2 = R_{jj}^2$. Auch wenn dieser Ansatz zunächst sinnvoll scheinen mag, werden wir in Kapitel \ref{application} anhand unseres Datensatzes zeigen, dass durch (\ref{zou_explained_variance}) keine korrekte Berechnung der erklärten Varianz erfolgt. Das Problem des Ansatzes liegt daran, dass der Zusammenhang zum Rekonstruktionsfehler nicht klar ist. Bei der klassischen Hauptkomponentenanalyse haben wir die erklärte Varianz mithilfe des Rekonstruktionsfehlers angeben können, was hier nicht mehr der Fall sein muss. 

Eine korrekte Methode wird von Camacho et al. eingeführt. Hierbei zerlegen wir die Varianz des Modells $\mat X = \mat Z \mat B^T + \mat E$ in zwei Teile.
\begin{align}
\label{camacho_explained_variance}
\norm{\mat X}_F^2 & = \norm{\mat Z \mat B^T + \mat E}_{F}^{2} \nonumber\\
& = \spur{\mat B \mat Z^T \mat Z \mat B^T} + \spur{\mat B \mat Z^T \mat E} + \spur{\mat E^T\mat Z \mat B^T} + \spur{\mat E^T \mat E} \nonumber\\
& = \spur{\mat B \mat Z^T \mat Z \mat B^T} + \spur{\mat E^T \mat E} \nonumber\\
& = \norm{\mat Z \mat B^T}_{F}^{2} + \norm{\mat E}_{F}^{2}
\end{align}
If deflation is performed in the score space $EP = P^T E^T = 0$.
Damit kann die Varianz eines Datensatzes in die Varianz der Rekonstruktion und der Residuen aufgeteilt werden. Ersterer Teil entspricht der erklärten Varianz unseres Modells und wird daher mit $\norm{\mat Z \mat B^T}_{F}^{2}$ berechnet.


%----------------------------------------------------------------------------------------
%	Wahl der Hyperparameter
%----------------------------------------------------------------------------------------

\section{Wahl der Hyperparameter}
\label{choice_of_tuning_parameters}

Bei der dünnbesetzten Hauptkomponentenanalyse sind mehrere Hyperparameter zu wählen. Dazu gehören die Anzahl an Hauptkomponenten $k$ und die Regularisierungsparameter $\lambda_{1,j}$ und $\lambda_2$. Im Folgenden möchte wir mögliche Vorgehensweise näher erläutern und eine Übersicht über mögliche Verfahren geben. 

Bevor man die Regularisierungsparameter festlegt ist es sinnvoll zunächst die Anzahl an Hauptkomponenten für das Modell zu bestimmen, da sich bei einer Änderung von $k$ alle Hauptkomponenten verändern können. Hierbei kann man analog zur klassischen Variante wie in Abschnitt \ref{selection_principal_components} beschrieben vorgehen. Man nutzt also zunächst die klassische Variante, um $k$ zu bestimmen.

Empirische Ergebnisse zeigen, dass die Ergebnisse sich mit Veränderung von $\lambda_2$ kaum ändern. Für einen $n > p$ Datensatz kann der Parameter auf Null gesetzt werden, da das Lasso-Defizit in diesem Fall nicht auftritt. In der Praxis wird $\lambda_2$ auf eine kleine positive Zahl in der Größenordnung $10^{-6}$ gesetzt, um mögliche Kollinearitätsprobleme zu vermeiden \cite{zou_sparsepca}. Falls $p \gg n$ werden wir eine Standardwahl von $\lambda_2$ in Kapitel \ref{implementation} treffen.

Komplizierter gestaltet sich eine Wahl von $\lambda_{1,j}$, welche die Balance zwischen Dünnbesetzung und Rekonstruktionsfehler regelt. Es wird keine Sturres Ausprobieren in Zou et al. Alternativ wird vorgeschlagen, den zweischrittigen Prozess aus REF zu verwenden, bei welcher man einen gesamten Lösungspfad für die $\lambda_{i,j}$ erhält. Bei der Verwendung dieses Ansatzes kommen allerdings unterschiedliche Lösungne nrua.s

Im Prinzip könnte man die $\lambda_{1,j}$ durch ein Kreuzvalidierungsverfahren bestimmen. Je nach Größe des Datensatzes kann dies aber sehr rechenintensiv sein, weshalb wir hier einen alternativen Ansatz beschreiben möchten. In der Literatur wird meist ein Bayes-Informationskriterium (BIC) angegeben, welches aber je nach Anwendung und Generalisierung verschieden formuliert wird. Zwei Varianten, welche auf \cite{hubert, allen} (\ref{bic_reconstruction}) und \cite{croux, guo} (\ref{bic_ratio}) zurückgehen, werden wir hier betrachten.
\begin{align}
\label{bic_reconstruction}
\operatorname{BIC}(\lambda_{1,j}) = \log\left(\frac{\norm{X - Z_j\beta_j^T}_{F}^{2}}{np}\right) + \operatorname{df}(\lambda_{1,j}) \frac{\log(np)}{np}
\end{align}
wobei $\operatorname{df}(\lambda_{1,j}) = \norm{\beta_j}_0$ die Anzahl von null verschiedener Einträge sind. Dabei steht $\operatorname{df}$ für \textit{degrees of freedom}, welches die Anzahl freier Parameter bzw. den Freiheitsgrad darstellt \cite{hastie_elements}. Klar erkennbar in (\ref{bic_reconstruction}) ist der Kompromiss zwischen Rekonstruktionsfehler der $j$-ten Hauptkomponente und der Dünnbesetzung durch $\operatorname{df}(\lambda_{1,j})$. Mit steigendem $\lambda_{1,j}$ wird der Rekonstruktionsfehler größer und die Anzahl freier Parameter geringer. Wir sind angehalten $\lambda_{1,j}$ zu finden, die eine Balance zwischen den beiden Termen ermöglicht.

Um nicht über $k$ Parameter optimieren zu müssen, kann man beispielsweise $\lambda_{1,j} = \lambda_1$ für alle $1 \leq j \leq k$ setzen. Eine weitere Möglichkeit wird in \cite{croux} beschrieben. Für $j>1$ sei $\mat{\tilde{B}}_{j-1}^{\perp}$ die Matrix, deren Spalten eine orthonormale Basis für das orthogonale Komplement für den durch $\beta_1, \ldots, \beta_{j-1}$ aufgespannten Raum sind. Wir bezeichnen $x_i^{(j-1)} = (\mat{\tilde{B}}_{j-1}^{\perp})^T x_i$ für $i = 1, \ldots, n$ und setzen $\lambda_{1,j} = \lambda_1\operatorname{Var}[\mat X^{(j)}]$, wobei $\mat X^{(j)}$ aus den auf das orthogonale Komplement der ersten $j-1$ Hauptachsen projizierten Daten $x_i^{(j-1)}$ besteht. Mithilfe dieses Ansatzes können wir eine vergleichbare Dünnbesetzung für alle Hauptachsen erhalten. Für die Wahl von $\lambda_1$ wird ein ähnliches BIC-Kriterium vorgeschlagen.
\begin{align}
\label{bic_ratio}
\operatorname{BIC}(\lambda_1) = \frac{\norm{\mat X - \mat X \mat B \mat A^T}_{F}^{2}}{\norm{\mat X - \mat X \mat V \mat V^T}_{F}^{2}} + \operatorname{df}(\lambda_1) \frac{\log(n)}{n}
\end{align}
Für die log-likelihood-Funktion wird in (\ref{bic_ratio}) das Verhältnis zwischen Rekonstruktionsfehler der dünnbesetzten und der klassischen Hauptkomponentenanalyse gewählt. Welche der beiden BIC-Kriterien genutzt werden sollte, kommt auf den Anwendungsfall und den Rechenaufwand an, welchen man bereit ist in Kauf zu nehmen.

Typischerweise wird für die Minimierung der BIC-Kriterien eine Rastersuche für $\lambda_1$ im Wertebereich $[0, \lambda_1^{max}]$ durchgeführt, wobei eine Wahl von $\lambda_1^{max}$ in Hauptachsen mit nur einem von Null verschiedenem Eintrag resultieren. Andere Verfahren wie die Zufallssuche, Bayessche oder gradienbasierte Optimierung sind an dieser Stelle denkbar, um ein passendes $\lambda_1$ zu finden. 

Wichtig zu erwähnen ist, dass BIC-Kriterien sich in $p \gg n$-Situationen schlecht verhalten.
The BIC suffers from two main limitations[5]
the above approximation is only valid for sample size n much larger than the number k of parameters in the model.
the BIC cannot handle complex collections of models as in the variable selection (or feature selection) problem in high-dimension. 


%----------------------------------------------------------------------------------------
%	Theoretische Aussagen
%----------------------------------------------------------------------------------------


\section{Theoretische Aussagen} 
\label{spca_theorems}


Zu Abschluss dieses Kapitels möchten wir uns mit theoretischen Aussagen zur dünnbesetzten Hauptkomponentenanalyse auseinandersetzen.  Von wesentlicher Bedeutung ist die Konsistenz der Methode im Vergleich zur klassischen Variante.

Durch ref werden wir sehen, dass das Sparse PCA Kriterium nur von der Kovarianzmatrix abhängt. Um eine Populationsversion der dünnbesetzen Hauptkomponete zu erhalten ersetzen wir $\mat X^T \mat X$ schlicht durch die Kovarianzmatrix $\mat \Sigma$.





recomputation necessary of principal components when changing number of pc's.

Alexandre d’Aspremont, Optimal Solutions for Sparse Principal Component Analysis
We then use the same relaxation to derive sufficient conditions for global optimality of a
solution, which can be tested in O(n
3
) per pattern. We discuss applications in subset selection and
sparse recovery and show on artificial examples and biological data that our algorithm does provide
globally optimal solutions in many cases.
% Main chapter title
\chapter{Implementierung}

% Chapter label
\label{implementation}

In diesem Kapitel werden wir einen Algorithmus beschreiben, der das Sparse PCA Kriterium (\ref{spca_criterion}) minimiert. Dazu werden wir uns zunächst mit dem allgemeinen Fall beschäftigen bevor wir den Fall $n \ll p$ genauer betrachten, um eine effiziente Berechnung zu garantieren. Zu Schluss dieses Kapitels werden wir uns genauer mit der Laufzeit des Algorithmus auseinandersetzen. 

\section{Numerische Lösung}
\label{spca_numerical_solution}

Zunächst möchten wir das Sparse PCA Kriterium erneut formulieren. Sei $\mat{A}_{p \times k} = [ \alpha_1, \ldots ,\alpha_k ]$ und $\mat{B}_{p \times k} = [ \beta_1, \ldots ,\beta_k ]$. Wie zuvor bezeichnen wir mit $x_i$ die $i$-te Zeile von $\mat X$. Dann lautet das Sparse PCA Kriterium

$$(\hat{\mat{A}}, \hat{\mat{B}}) = \argmin_{\mat{A}, \mat{B}} \sum_{i=1}^{n} \norm{x_i - \mat{A}\mat{B}^Tx_i}_2^2 + \lambda \sum_{j=1}^{k}\norm{\beta_j}_2^2 + \sum_{j=1}^k \lambda_{1,j} \norm{\beta_j}_1$$

$$\text{unter der Nebenbedingung, dass } \mat{A}^T\mat{A} = I_{k \times k}$$

Durch die Einführung der Matrix $\mat B$ in das Kriterium in Kapitel \ref{sparse_pca} wird über die beiden Matrizen $\mat A$ und $\mat B$ minimiert. Man kann zeigen, dass es sich bei (...) um ein nicht-konvexes Optimierungsproblem handelt. (CITE) Allerdings stellt sich heraus, dass das Problem konvex für festes $\mat A$ bzw. festes $\mat B$ ist. (Überprüfung!!!) Daher liegt es nahe einen alternierenden Ansatz zu wählen, um das Problem numerisch zu lösen. Wir betrachten im Folgenden also zwei Optimierungsprobleme.

$\mat B$ \textbf{gegeben} $\mat A$:

Wir wenden uns zunächst der Verlustfunktion zu. Hierfür sei $\mat A_{\perp} \in \mathbb{R}^{p \times (p-k)}$ eine orthonormale Matrix, so dass $[\mat A ; \mat A_{\perp}]$ $p \times p$ orthonormal ist. Dann gilt

\begin{align*}
\sum_{i=1}^{n} \norm{x_i - \mat{A}\mat{B}^Tx_i}_2^2 & = \norm{\mat X - \mat X \mat B \mat A^T}_F^{2}\\
& = \norm{\mat X \mat A_{\perp}}_F^2 + \norm{\mat X \mat A - \mat X \mat B}_F^2\\
& = \norm{\mat X \mat A_{\perp}}_F^2 + \sum_{j=1}^{k}\norm{\mat X \alpha_j - \mat X\beta_j}_2^2
\end{align*}

Wir setzen $Y_j^* = \mat X \alpha_j$ für alle $1 \leq j \leq n$. Somit reduziert sich (...) für fixes $\mat A$ auf das Lösen von $k$ elastic net Problemen
\begin{align}
\label{sub_problem_enet}
\hat{\beta}_j = \argmin_{\beta_j} \norm{Y_j^* - \mat X \beta_j}_2^2 + \lambda \norm{\beta_j}^2 + \lambda_{1,j}\norm{\beta_j}_1
\end{align}
In Kapitel \ref{fundamentals} haben wir uns bereits mit elastic nets beschäftigt und einen effizienten Algorithmus zur Lösung dieser präsentiert.

$\mat A$ \textbf{gegeben} $\mat B$:

Fixieren wir die Matrix $\mat B$, so können wir uns auf das Minimieren der Zielfunktion $\sum_{i=1}^{n} \norm{x_i - \mat{A}\mat{B}^Tx_i}_2^2 = \norm{\mat X - \mat X \mat B \mat A^T}_F^{2}$ beschränken, da die Bedingungen an $\beta_j$ nicht von Relevanz beim Optimieren über $\mat A$ sind. Somit ergibt sich 
\begin{align}
\label{sub_problem_procrustes}
\begin{split}
\hat{\mat A} = \argmin_{\mat A}\norm{\mat X - \mat X \mat B \mat A^T}_F^2\\
\text{unter der Nebenbedingung, dass } \mat A \mat A^T = \mat I_{k \times k}
\end{split}
\end{align}

Für dieses Optimierungsproblem lässt sich eine explizite Lösung angeben, da es eine Form von Procrustes Rotationsproblem ist, welches wir in Theorem REF beschrieben haben. Sei daher 
$$(\mat X^T \mat X) \mat B = \mat U \mat D \mat V^T$$ eine Singulärwertzerlegung. Dann ist $\hat{\mat A} = \mat U \mat V^T$.

Es ist sinnvoll zu erwähnen, dass für die Lösung beider Teilprobleme nur die Gram-Matrix $\mat X^T \mat X$ bekannt sein muss. Dies erleichtert die Berechnung.

\section{Algorithmus}

Durch die Vorarbeit im vorangegangenem Abschnitt können wir einen effizienten Algorithmus zur Lösung des Sparse PCA Kriteriums angeben. Zuerst initialisieren wir $\mat A$ mit den ersten $k$ Hauptachsen. Anschließend minimieren wir abwechselnd über $\mat B$ gegeben $\mat A$ und $\mat A$ gegeben $\mat B$ solange bis ein geeignetes Konvergenzkriterium erfüllt ist oder wir eine maximale Anzahl an Iterationen erreicht haben. Durch abschließende Normalisierung der Spalten von $\mat B$ erhalten wir die dünnbesetzten Hauptachsen. Eine Übersicht haben wir in Algorithmus \ref{spca_algorithm} erstellt.

\begin{algorithm}[tbh]
    \caption{Sparse Principal Component Analysis}
    \label{spca_algorithm}
    \begin{algorithmic}[1]
        \Procedure{SPCA}{$\mat A, \mat B$}
        	\State $\mat A \gets \mat V[,1 \colon k]$, die ersten $k$ Hauptachsen
            \While{nicht konvergiert}
                \State Gegeben festes $\mat A = [\alpha_1, \ldots, \alpha_k]$, löse das elastic net Problem
                $$\hat{\beta}_j = \argmin_{\beta_j} \norm{\mat X \alpha_j - \mat X \beta_j}^{2} + \lambda \norm{\beta_j}^2 + \lambda_{1,j}\norm{\beta_j}_{1} \quad \text{für } j = 1, \ldots, k$$
                \State Gegeben festes $\mat B = [\beta_1, \ldots, \beta_k]$, berechne die Singulärwerzerlegung von $$\mat X^T \mat X \mat B = \mat U \mat D \mat V^T$$
                $$\mat A \gets \mat U \mat V^T$$
            \EndWhile
            \State $\hat{V}_j \gets \frac{\beta_j}{\norm{\beta_j}}$ for $j = 1, \ldots, k$
        \EndProcedure
    \end{algorithmic}
\end{algorithm} 

Es stellt sich nun die Frage nach einem passendem Abbruchkriterium. Da für uns am Schluss des Algorithmus nur die dünnbesetzten Hauptachsen relevant sind, liegt es nahe ein Konvergenzkriterium für $\mat B$ zu wählen. Zou et al. brechen in ihrer Implementierung die Iteration ab, falls
$$\max_{\substack{1 \leq i \leq p \\ 1 \leq j \leq k}} \left| \frac{beta_{ij}^{(l+1)}}{\norm{\beta_i}} - \frac{beta_{ij}^{(l)}}{\norm{\beta_i}} \right| < \epsilon$$
wobei $\beta_{ij}^{(l)}$ der $j$-te Eintrag der dünnbesetzten Hauptachse $\beta_i$ in der $l$-ten Iteration ist. Sobald also die Änderung in $\mat B^{(l)}$ klein genug ist, kann die while-Schleife abgebrochen werden. Um die Laufzeit des Algorithmus zu beschränken, ist es sinnvoll ein zusätzliches Abbruchkriterium zu definieren. So werden wir bei Anwendung des Algorithmus eine maximale Anzahl an Iterationen $l_{max}$ festlegen, die nicht überschritten werden darf.

\section{Numerische Lösung im Fall p >> n}

\section{Laufzeitvergleich}

\section{Eigene Implementierung in Python}

Momentan existieren Implementierungen der dünnbesetzten Hauptkomponentenanalyse in R und Python. Zou et al. stellen das elasticnet package mit einer spca-Funktion, welche auf ihrem Ansatz beruht in der Programmiersprache R zur Verfügung. Für die Lösung des Subproblems (\ref{sub_problem_enet}) wird der LARS-EN Algorithmus gewählt, welche eine Erweiterung des LARS-Algorithmus für Elastic Nets ist \cite{zou_elasticnet}. Dagegen bietet scikit-learn eine SparsePCA-Variante in Python, welche auf Jennaton et al. in \cite{jennaton} zurückgeht und einen leicht anderen Ansatz verfolgt.

Um ein genaues Verständnis der Ergebnisse zu garantieren, haben wir uns dazu entschieden, eine eigene Implementierung in Python vorzunehmen, die auf dem Ansatz von Zou et al. beruht. Es wurde kritisch überprüft, dass der von uns implementierte Code dieselben Ergebnissen wie der spca-Algorithmus aus dem elasticnet package liefert. Dazu haben wir den Pitprops Datensatz aus \cite{zou_sparsepca}, welcher oft als Benchmark genutzt wird, und zusätzlich den eigenen Datensatz, welchen wir in Kapitel \ref{application} beschreiben, verwendet. Gegenüber der Implementierung im elasticnet package haben wir zwei entscheidende Änderungen vorgenommen, welche die Laufzeit in der Praxis deutlich verkürzen. Statt das Subproblem (\ref{sub_problem_enet}) mit LARS-EN zu lösen, wählen wir ein Koordinaten-Abstiegsverfahren.
Des Weiteren wird in der Implementierung von Zou et al. die Gram-Matrix $\mat X^T \mat X$ vorab berechnet, um für das Subproblem (\ref{sub_problem_procrustes}) nur eine Multiplikation pro Iteration $(\mat X^T \mat X) \mat B$ durchführen zu müssen. Da die Gram-Matrix aber in $\mathbb{R}^{p \times p}$ liegt, ist es für den Fall $p \gg n$ sinnvoller, $\mat X^T (\mat X \mat B)$ naiv zu berechnen, damit keine $p \times p$-Matrix zwischengespeichert werden muss. Dies ermöglicht für unseren hochdimensionalen Datensatz eine Laufzeit, die etwa um den Faktor $4$ besser ist.

Eine weitere kleine Änderung haben wir in der Aufrufstruktur der Methode vorgenommen, um die Notation mit der ElasticNet-Klasse in scikit-learn zu vereinheitlichen. Ähnlich wie in Abschnitt \ref{comparison_linear_models} haben wir eine Reparametrisierung vorgenommen
\begin{align}
 \alpha = \frac{2\lambda_2 + \lambda_1}{2n} \quad \text{und} \quad \gamma = \frac{\lambda_1}{2\lambda_2 + \lambda_1}
\end{align}
wobei $\alpha$ die Stärke der Bestrafung und $\gamma$ das Verhältnis der beiden Strafterme beschreibt.
% Main chapter title
\chapter{Anwendung}

% Chapter Label
\label{application}

In diesem Kapitel beschäftigen wir uns mit der Anwendung der dünnbesetzten Hauptkomponentenanalyse auf Frequenzdaten einer Mühle. Dafür stehen uns Zeitreihen von Beschleunigungssensoren und Mikrofonen zur Verfügung, welche an der Maschine angebracht sind, um die Vibration bzw. die Akustik zu messen. Wir sind interessiert daran herauszufinden, ob sich mithilfe der Zeitreihen Aussagen über die Partikelgröße des Materials treffen lassen. Aufgrund der geringen Beobachtungszahl des Datensatzes sind viele überwachte Lernverfahren in diesem Zusammenhang unbrauchbar. Im Zuge einer explorativen Analyse kann daher eine Dimensionsreduktion sinnvoll sein. Dabei sind wir nicht unbedingt an der niedrigdimensionalen Repräsentation der Daten interessiert, sondern viel mehr an der Herausfilterung wichtiger Frequenzen. Idealerweise können wir die Frequenzen dem Material oder der Maschine zuordnen, so dass wir zwischen Material- und Maschineneigenschaften unterscheiden können. Die Möglichkeit der Interpretation und die Konsistenz in hochdimensionalen Fällen waren Anlass für die Verwendung der dünnbesetzten Hauptkomponentenanalyse in diesem Zusammenhang. 

Zunächst werden wir den Datensatz in Abschnitt \ref{data_set} näher beschreiben und Vorverarbeitungsschritte in \ref{preprocessing} erläutern. Nachdem wir in \ref{application_frequency_data} erklären, welche Experimente durchgeführt worden sind, werden wir uns den Ergebnissen in \ref{evaluation} zuwenden. Zu einer detaillierten Auswertung gehört sowohl ein Vergleich mit der klassischen Hauptkomponentenanalyse, als auch eine Analyse des Verhalten des Algorithmus. Dazu werden wir die Wahl der Hyperparameter, die Laufzeit und die Konvergenz thematisieren. Zum Abschluss dieses Kapitels werden wir die Korrektheit der von Camacho et al. \cite{camacho} berechneten Varianzen empirisch validieren. Ob eine derartige Methode für diesen Datensatz ein sinnvoller Ansatz war, diskutieren wir in Kapitel \ref{conclusion}.


%----------------------------------------------------------------------------------------
%	Beschreibung des Datensatzes
%----------------------------------------------------------------------------------------


\section{Beschreibung des Datensatzes}
\label{data_set}

Wir verfügen über Zeitreihen verschiedener Sensoren, 
die an unterschiedlichen Positionen einer Mühle angebracht sind. Insgesamt wurden $n \approx 30$ Messungen bei laufendem Mahlprozess gestartet. Bei jeder dieser wurde dasselbe Material verwendet und ein festgelegter Zeitraum betrachtet. Um eine Trennung von Maschine und Material in den Daten zu ermöglichen, wurden Messungen sowohl mit als auch ohne Material durchgeführt. Des Weiteren wurden verschiedene Betriebszustände der Maschine variiert, damit ein möglicher Zusammenhang mit dem Mahlergebnis hergestellt werden kann. Durch eine hohe Abtastrate haben wir es mit einem hochdimensionalen Datensatz zu tun. Für jeden angebrachten Sensor erhalten wir Auslenkungswerte für $p \approx 5,000,000$ Zeitpunkte. Mit einer geringen Beobachtungszahl, wobei Messungen mit Material mehrmals aufgezeichnet worden sind, sehen wir uns mit einer \textit{high dimension low sample size} Situation konfrontiert. 

\section{Vorverarbeitung der Daten}
\label{preprocessing}

Vor der Anwendung der dünnbesetzten Hauptkomponentenanalyse haben wir einige Vorverarbeitungsschritte vorgenommen. Da die Messungen zu zufälligen Zeitpunkten bei laufendem Mahlprozess gestartet worden sind, können einzelne Zeitpunkte nicht direkt miteinander verglichen werden. Mit einer Fouriertransformation der Daten können wir anstatt auf der Zeitachse auf den Frequenzen arbeiten, welche uns tiefere Einblicke ermöglichen. Auf diesem Wege sind sowohl Unterschiede zwischen Messungen, als auch mögliche Rauscheffekte leichter zu erkennen. In Abbildung \ref{fft_example} zeigen wir das Ergebnis einer Fouriertransformation beispielhaft für eine Messung eines Beschleunigungssensors.

\begin{figure}
\centering
\includegraphics[width=\textwidth]{figures/Signal_5_fft_example_new.jpg}
\caption{Signal 5 fft example}
\label{fft_example}
\end{figure}

Es wird sich zeigen, dass der Algorithmus für die dünnbesetzte Hauptkomponentenanalyse sehr rechenintensiv sein kann. Daher haben wir uns entschieden, nur einen Teil der ursprünglichen Zeitreihe zu verwenden. Mithilfe eines Blicks auf das Spektrogramm in Abbildung \ref{spectrogram} erkennen wir, dass sich die Frequenzen über den Messzeitraum kaum verändern. Dies ist darauf zurückzuführen, dass der Maschine konstant Material zugeführt wird. Daher können wir die Dimension um einen Faktor $100$ reduzieren ohne wichtige Informationen zu verlieren. Des Weiteren wurden Teile der Frequenzen, welche außerhalb des Frequenzbereichs des jeweiligen Sensors liegen, abgeschnitten. Somit verbleiben wir mit $p \approx 15,000$ Variablen. Um die Varianzen der Frequenzen vergleichbarer zu machen, haben wir die Daten ähnlich wie bei der klassischen Hauptkomponentenanalyse zentriert.

\begin{figure}
\centering
\includegraphics[width=\textwidth]{figures/Signal_5_time_frequency_change.jpg}
\caption{Signal 5 time frequency change}
\label{spectrogram}
\end{figure}

%----------------------------------------------------------------------------------------
%	Anwedung auf Frequenzdaten
%----------------------------------------------------------------------------------------


\section{Anwendung auf Frequenzdaten}
\label{application_frequency_data}

Unsere Implementierung ermöglicht eine Wahl verschiedener Parameter bezüglich des Modells bzw. des Algorithmus. Für eine Beschränkung der Laufzeit setzen wir eine maximale Anzahl an Iterationen von $500$ und eine Toleranz von $10^{-4}$. Falls nach $500$ Iterationen die vorgegebene Toleranz noch immer nicht erreicht ist, werden wir dies im Folgenden kenntlich machen. Ein Modellparameter, den es zu wählen gilt, ist die Anzahl zu berechnender Hauptkomponenten. Wie bereits in \ref{choice_of_tuning_parameters} beschrieben, wird dazu meist das Ergebnis einer klassischen Hauptkomponentenanalyse verwendet. Hierbei hat sich gezeigt, dass $2$ Hauptkomponenten für die meisten Sensoren ausreichen, um einen Großteil des Datensatzes zu erklären. Zu Analysezwecken haben wir uns entschieden einen Durchlauf mit $2$ und einen mit $10$ Hauptkomponenten zu starten. 

Interessanter ist die Wahl der Hyperparameter $\lambda$ und $\alpha$, welche wesentlichen Einfluss auf die Ergebnisse besitzen. Auch wenn prinzipiell die Möglichkeit besteht, differenzierte Bestrafungen je Hauptkomponente $\lambda_{1,j}$ auszuwählen, beschränken wir uns der Einfachheit halber auf eine einheitliche Bestrafung. Für die Wahl von $\lambda$ haben wir sowohl mehrere Werte ausprobiert, als auch eine Rastersuche bezüglich der in \ref{choice_of_tuning_parameters} beschriebenen BIC-Kriterien durchgeführt. Dafür verwenden wir auf einer log-Skala gleichverteilte Werte im Bereich zwischen $10^{-7}$ und $10^0$. Dagegen wählen wir für das Verhältnis zwischen Lasso und Ridge-Bestrafung die Werte $[0.1,\, 0.5,\, 0.7,\, 0.9,\, 0.95,\, 0.99,\, 1]$. Mithilfe des BIC-Kriteriums können wir zeitgleich über $\lambda$ und $\alpha$ optimieren. Es hat sich in den Anwendungen gezeigt, dass eine geeignete Liste für $\alpha$ mehr Werte nahe $1$ hat, da sich dort die größten Änderungen ergeben. Damit stärken wir den Lasso-Strafterm im Vergleich zum Ridge-Strafterm.

Exemplarisch werden wir uns nun mit einem der Beschleunigungssensoren weiter beschäftigen. An dieser Stelle möchten wir erwähnen, dass wir die Sensoren getrennt betrachten, d.h. die dünnbesetzte Hauptkomponentenanalyse auf die Sensoren einzeln anwenden. Dies ist aufgrund der Unterschiedlichkeit der Sensoren sinnvoll. Wir werden nun ausgewählte Ergebnisse vorstellen. 


%----------------------------------------------------------------------------------------
%	Auswertung der Ergebnisse
%----------------------------------------------------------------------------------------


\section{Auswertung der Ergebnisse}
\label{evaluation}

Im Rahmen dieser Arbeit können wir nun einen begrenzten Teil der Ergebnisse präsentieren. Ziel wird es sein, wesentliche Ergebnisse und Effekte des Algorithmus zu erläutern, um ein transparentes Modell zu entwickeln.


%----------------------------------------------------------------------------------------
%	Klassische Analyse der Hauptachsen
%----------------------------------------------------------------------------------------


\subsection{Klassische Analyse der Hauptachsen}

Zunächst wollen wir eine klassische Analyse durchführen, wie sie oft für ein derartiges Verfahren gemacht wird. Um einen Vergleich zu ermöglichen, haben wir sowohl die klassische als auch die dünnbesetzte Hauptkomponentenanalyse durchgeführt. 

Für einen ersten Einblick in die Ergebnisse betrachten wir Abbildung \ref{sparse_pca_classical_analysis_pc_graph} bzw. \ref{sparse_pca_classical_analysis_sparse_pc_graph}, in welcher die ersten beiden Hauptkomponenten gegeneinander aufgezeichnet sind. Dies ist die Darstellung der Daten bezüglich der neu gefundenen Variablen. Zuerst wenden wir uns den Ergebnissen der klassischen Hauptkomponentenanalyse zu. Man sieht schnell, dass sich vier Gruppierungen ergeben. Durch die Transformation sehen wir eine deutliche Trennung zwischen Messungen mit und ohne Material, welche im Bild durch verschiedene Farben markiert sind. Des Weiteren sind die Messungen mit bzw. ohne Material in zwei Untergruppen geteilt. Die Unterschiede sind auf verschiedene Betriebszustände der Maschine zurückzuführen, auf welche wir hier nicht genauer eingehen können. Vergleichen wir dieses Ergebnis mit der dünnbesetzten Variante in Abbildung \ref{sparse_pca_classical_analysis_sparse_pc_graph}, erkennen wir viele Gemeinsamkeiten. Es treten dieselben Gruppierungen wie zuvor auf, so dass noch immer eine Trennung der Messungen bezüglich der Befüllung möglich ist. Ein kleinerer Unterschied ist in der zweiten Hauptkomponenten zu sehen, da bei der dünnbesetzten Variante drei Messungen noch stärker separiert werden.

Nun gilt es, das Zustandekommen dieses Bildes zu erklären. Wir wollen verstehen, warum wir eine Trennung zwischen Messungen bezüglich der Befüllung sehen. Vor allem interessieren wir uns dafür, welche Frequenzen für diesen Unterschied verantwortlich sind. Zu diesem Zweck betrachten wir Abbildung \ref{sparse_pca_classical_analysis_principal_axis} bzw. \ref{sparse_pca_classical_analysis_sparse_principal_axis}, in welcher die Hauptachsen der beiden Varianten zu sehen sind. Anhand der Hauptachsen können wir erkennen, welche Frequenzen eine entscheidende Rolle bei der Erhaltung der maximalen Varianz spielen. Demnach sind die größten Unterschiede im Datensatz auf die Frequenzen mit der größten Amplitude zurückzuführen. Auch wenn es in den niederen Frequenzen nicht leicht zu erkennen ist, sind bei der klassischen Hauptachse alle $\approx 15,000$ Einträge von Null verschieden. Eine Erklärung der beschriebenen Effekte ist hier quasi unmöglich. Wir können lediglich aussagen, dass die höheren Frequenzen in einer gewissen Weise wichtig sind. Genau hier liegt das Problem der klassischen Variante, denn eine Interpretation der Hauptachsen ist selten aufschlussreich. Es fließen einfach zu viele Koeffizienten in das Modell ein. Dagegen reduziert sich bei der dünnbesetzten Variante die Anzahl von Null verschiedener Einträge auf $\approx 40$ für $\lambda = 0.1$ und $\alpha = 0.1$. Durch die eingeschränkte Modellkomplexität erkennen wir drei Peaks im Frequenzspektrum. Im Wesentlichen sind also nur viel weniger Frequenzen für die beschriebene Trennung verantwortlich. Daher können wir folgern, dass genau diese Frequenzen durch die Maschine verursacht werden. Mit diesem Verfahren ist es also möglich Frequenzen zu identifizieren, welche eine klare Bedeutung im Kontext besitzen.

Zuletzt betrachtet man Abbildung \ref{sparse_pca_classical_analysis_scree_plot}, welche einen sog. \textit{scree plot} zeigt und oft als Bewertungsmittel von Dimensionsreduktionsverfahren dient. Mithilfe dieses Bildes kann man einsehen, welchen Anteil der Varianz des Datensatzes mit der niedrigdimensionalen Repräsentation erklärt wird. Zur Verdeutlichung der Effekte haben wir $10$ Hauptkomponenten berechnet. Es zeichnet sich ein klarer Varianzverlust je Hauptkomponente ab, wenn wir die dünnbesetzte Hauptkomponentenanalyse nutzen. Hier erhalten wir nur $25$\% der Information des Datensatz, während es im Gegensatz dazu $95$\% bei der klassischen Variante sind. Besonders deutlich wird der Kompromiss, den wir eingehen müssen. Dadurch, dass unsere Hauptachsen einfacher zu interpretieren sind, verlieren wir an erklärter Varianz. Ein geeignetes Maß zwischen Modellkomplexität und Rekonstruktionsfehler zu finden, kann von der Anwendung und der Intention abhängen. In unserem Fall ist die Möglichkeit einer Interpretation der Hauptachsen deutlich wichtiger als die Varianzerhaltung. Wir haben durch Abbildung \ref{sparse_pca_classical_analysis_pc_graph} bzw. \ref{sparse_pca_classical_analysis_sparse_pc_graph} gesehen, dass uns durch einen Rückgang der Varianz nicht zwangsläufig Informationen verloren gehen müssen. Wir konnten trotz geringer Varianz ein ähnliches Bild mit denselben Gruppierungen erzielen.

\begin{figure}
\centering
\begin{subfigure}{0.45\textwidth}
\centering
\includegraphics[width = 0.9\textwidth]{figures/Signal_5_pc_graph.jpg}
\caption{Signal 5 PC Graph}
\label{sparse_pca_classical_analysis_pc_graph}
\end{subfigure}
%
\begin{subfigure}{0.45\textwidth}
\centering
\includegraphics[width = 0.9\textwidth]{figures/Signal_5_sparse_pc_graph.jpg}
\caption{Signal 5 Sparse PC Graph}
\label{sparse_pca_classical_analysis_sparse_pc_graph}
\end{subfigure}
%
\begin{subfigure}{0.9\textwidth}
\centering
\includegraphics[width=\textwidth]{figures/Signal_5_principal_axis.jpg}
\caption{Signal 5 principal axis}
\label{sparse_pca_classical_analysis_principal_axis}
\end{subfigure}
%
\begin{subfigure}{0.9\textwidth}
\centering
\includegraphics[width=\textwidth]{figures/Signal_5_sparse_principal_axis.jpg}
\caption{Signal 5 sparse principal axis}
\label{sparse_pca_classical_analysis_sparse_principal_axis}
\end{subfigure}
%
\begin{subfigure}{0.9\textwidth}
\centering
\includegraphics[width = \textwidth]{figures/Signal_5_scree_plot_10.jpg}
\caption{Signal 5 Scree Plot}
\label{sparse_pca_classical_analysis_scree_plot}
\end{subfigure}
\caption{Vergleich der klassischen mit der dünnbesetzten Hauptkomponentenanalyse für $\lambda=10^{-4}$ und $\alpha = 0.95$.}
\label{sparse_pca_classical_analysis}
\end{figure}


%----------------------------------------------------------------------------------------
%	Wahl der Hyperparameter
%----------------------------------------------------------------------------------------


\subsection{Wahl der Hyperparameter}

Im vorangegangenem Abschnitt haben wir beispielhaft gezeigt, wie die Ergebnisse einer dünnbesetzten Hauptkomponentenanalyse zu interpretieren sind. Für die Analyse haben wir bestimmte Werte von $\alpha$ und $\lambda$ vorgegeben, die möglichst gute Ergebnisse erzielen. Es stellt sich jedoch die Frage, wie sich die einzelnen Hauptachsen und Hauptkomponenten verändern, falls wir andere Werte für die Hyperparameter wählen. Zu diesem Zweck wenden wir uns Abbildung \ref{results_parameter_benchmark} zu. Hier haben wir die Anzahl an Hauptkomponenten fixiert und versucht, die Effekte in Abhängigkeit von $\lambda$ für einen akustischen Sensor zu beschreiben.

Abbildung \ref{results_parameter_benchmark_degrees_of_freedom} zeigt wie sich $\operatorname{df}(\lambda)$, also die Anzahl von Null verschiedener Einträge in den Hauptachsen, bei Veränderung von $\lambda$ bzw. $\alpha$ verhält. Klar erkennbar ist der relativ gleichmäßige Abfall der Freiheitsgrade, d.h. für wachsendes $\lambda$ werden unsere Hauptachsen zunehmend dünnbesetzt. Dies entspricht unseren Erwartungen aus Kapitel \ref{sparse_pca}. Verschieben wir das Verhältnis der Bestrafung von der $\ell_1$ zur $\ell_2$-Norm, sprich kleineres $\alpha$, steigt die Anzahl der Freiheitsgrade. Dies bestätigt, dass die $\ell_1$-Norm im Gegensatz zur $\ell_2$-Norm eine Dünnbesetzung hervorruft. Ab einem gewissem Punkt $\lambda \approx 10^{-2}$ für $\alpha > 0.5$ bzw. $\lambda \approx 10^{-1}$ für $\alpha = 0.1$ ist die Bestrafung zu stark, so dass die Hauptachsen dem Nullvektor entsprechen und keine Anpassung an den Datensatz mehr stattfindet.

Interessant ist nun, wie sich die erklärte Varianz des Datensatzes im Vergleich verhält, welche in Abbildung \ref{results_parameter_benchmark_explained_variance} zu sehen ist.  Auffällig ist, dass sich für $\lambda$ im Bereich von $[10^{-7}, 10^{-3}]$ kaum Änderungen in der Varianz ergeben. In diesem Bereich sind wir nur leicht unter dem Niveau der klassischen Variante. Im Umkehrschluss können wir aufgrund der kontinuierlichen Stagnation der Freiheitsgrade die Modellkomplexität verringern, aber zeitgleich den Rekonstruktionsfehler auf konstantem Niveau halten. Erst nahe $\lambda \approx 10^{-2}$ für $\alpha > 0.5$ bzw. bei $\lambda \approx 10^{-1}$ für $\alpha = 0.1$ zeichnet sich ein deutlicher Einbruch ab. Dieser ist dadurch zu erklären, dass die Hauptachsen dann dem Nullvektor entsprechen. Aus der Kombination der beiden Abbildungen können wir schließen, dass nur wenige Frequenzen wirklich wichtig zur Erklärung der Varianz des Datensatzes notwendig sind.

Um eine automatisierte Wahl von $\lambda$ und $\alpha$ zu ermöglichen, haben wir in Abschnitt \ref{choice_of_tuning_parameters} Vorgehensweisen mithilfe eines BIC-Kriteriums beschrieben. Eine Anwendung des Kriteriums nach \cite{croux, guo} befindet sich in Abbildung \ref{results_parameter_benchmark_bic}. Hier zeichnet sich ein Minimum im Bereich von $\lambda \in [10^{-4}, 10^{-3}]$ für $\alpha > 0.5$ bzw. nahe $\lambda \approx 10^{-2}$ für $\alpha = 0.1$ ab, welches wir nach obiger Analyse erwarten konnten. Es wird ein Punkt gewählt, an welchem die erklärte Varianz gerade noch auf sehr hohem Niveau, aber die Modellkomplexität gering ist. Letztere Abbildung ist also eine Kombination der Erkenntnisse und kann genutzt werden, um eine Balance zwischen Dünnbesetzung und erklärter Varianz zu finden. An dieser Stelle möchten wir erwähnen, dass die Resultate des BIC-Kriteriums nicht für alle Sensoren zufriedenstellend waren. Genauer gesagt sehen wir in manchen Fällen, dass die Gewichtung zwischen Modellkomplexität und Rekonstruktionsfehler nicht passend gewählt ist. Dies kann daran liegen, dass BIC-Kriterien in hochdimensionalen Fällen versagen können \cite{giraud}. Für unsere Zwecke haben wir daher die betroffenen Sensoren mit manuellen Gewichten korrigiert.

\begin{figure}
\centering
\begin{subfigure}{0.9\textwidth}
\includegraphics[width=\textwidth]{figures/Signal_0_degrees_of_freedom.jpg}
\caption{Anzahl Freiheitsgrade in Abhängigkeit von $\lambda$.}
\label{results_parameter_benchmark_degrees_of_freedom}
\end{subfigure}
%
\begin{subfigure}{0.9\textwidth}
\includegraphics[width=\textwidth]{figures/Signal_0_explained_variances.jpg}
\caption{Erklärte Varianz des Datensatzes in Abhängigkeit von $\lambda$.}
\label{results_parameter_benchmark_explained_variance}
\end{subfigure}
%
\begin{subfigure}{0.9\textwidth}
\includegraphics[width=\textwidth]{figures/Signal_0_bic.jpg}
\caption{Wahl der Hyperparameter mithilfe eines minimalen BIC-Kriteriums.}
\label{results_parameter_benchmark_bic}
\end{subfigure}
\caption{Die Abbildung zeigt wie sich eine Wahl der Hyperparameter $\alpha$ und $\lambda$ mithilfe eines BIC-Kriterium gestalten kann. Um Erkenntnisse über die Entstehung der BIC-Abbildung zu gewinnen, werden zusätzlich die beiden Komponenten Anzahl Freiheitsgrade und erklärte Varianz gezeigt. Zu beachten ist die logarithmische Skala für $\lambda$.}
\label{results_parameter_benchmark}
\end{figure}


%----------------------------------------------------------------------------------------
%	Verhalten des Algorithmus
%----------------------------------------------------------------------------------------


\subsection{Verhalten des Algorithmus}

Im Zuge dieser Arbeit möchten wir nicht nur die Ergebnisse einiger Experimente, sondern auch das Verhalten des Algorithmus an sich beschreiben. Dafür sehen wir uns verschiedene Größen wie Laufzeit, Anzahl Iterationen und Toleranz an. Um ein Gefühl für die dünnbesetzte Hauptkomponentenanalyse zu bekommen, haben wir einen Überblick dieser Größen in Tabelle \ref{algorithm_analysis_table} erstellt. Interessant dabei ist vor allem die Abhängigkeit vom Hyperparameter $\lambda$. Je kleiner wir die Bestrafung wählen, desto länger dauert der Algorithmus und desto mehr Iterationen werden benötigt. Da bei kleinem $\lambda$ mehr von Null verschiedene Einträge $j$ in den Hauptachsen erlaubt werden, stimmen diese Beobachtung mit unseren Erwartungen überein. Der Anstieg der Zahl der Iterationen ist dadurch zu erklären, dass sich das Konvergenzkriterium auf alle Koeffizienten bezieht. Somit müssen bei einer Erhöhung von $j$ effektiv mehr Bedingungen erfüllt werden. Bezüglich der Laufzeit stimmen die Ergebnisse mit der Komplexität des Algorithmus von Abschnitt \ref{complexity} überein. Diese wird dabei nicht überraschend von der Lösung der Elastic Nets dominiert. Hingegen sind die Kosten für das Minimieren über $\mat A$ unabhängig von den Hyperparametern und im Wesentlichen durch eine Singulärwertzerlegung bestimmt, welche für diesen Datensatz in einem Bruchteil einer Sekunde gelöst werden kann. Ein Blick auf Tabelle \ref{algorithm_analysis_profiler} zeigt die Ergebnisse eines Profilers und bestätigt unsere Behauptungen.

\setlength{\tabcolsep}{10pt}
\begin{table}
\centering
\begin{tabular}{rrrr}
\boldmath$\lambda$ & \textbf{Laufzeit} & \textbf{Iterationen} & \textbf{Toleranz}\\\hline\addlinespace
$10^{-6}$ & $65235.76$ sec & $>500$ & $2.271780 \cdot 10^{-1}$\\
$10^{-5}$ & $9158.24$ sec & $>500$ & $4.132993 \cdot 10^{-4}$\\
$10^{-4}$ & $777.36$ sec & $99$ & $4.233382 \cdot 10^{-5}$\\
$10^{-3}$ & $201.49$ sec & $54$ & $5.531939 \cdot 10^{-5}$\\
$10^{-2}$ & $20.19$ sec & $5$ & $0$\\
$10^{-1}$ & $4.65$ sec & $1$ & $0$\\
\end{tabular}
\caption{Die Tabelle gibt einen Überblick über die Veränderung von Laufzeit, Iteration und Toleranz in Abhängigkeit von $\lambda$. Dabei wurde $\alpha=0.5$ und die Anzahl an Hauptkomponenten $k=10$ fest gewählt.\\Gerechnet wurde auf einem Intel Xeon Gold 6130F@2.10GHz.}
\label{algorithm_analysis_table}
\end{table}

\begin{table}
\centering
\begin{tabular}{rrrrr}
\textbf{ncalls} & \textbf{tottime} & \textbf{percall} & \textbf{cumtime} & \textbf{filename:lineno(function)}\\\hline\addlinespace
$5010$ & $4892.863$ & $0.977$ & $4896.461$ & \_coordinate\_descent.py:266(enet\_path)\\
$26075$ & $4.948$ & $0.000$ & $4.948$ & \{method 'reduce' of 'numpy.ufunc' objects\}\\
$6524$ & $1.753$ & $0.000$ & $1.753$ & \{method 'dot' of 'numpy.ndarray' objects\}\\
$76211$ & $1.749$ & $0.000$ & $1.749$ & \{built-in method numpy.array\}\\
$502$ & $0.619$ & $0.001$ & $0.631$ & linalg.py:1458(svd)\\
$5010$ & $0.412$ & $0.000$ & $0.942$ & validation.py:800(check\_random\_state)\\
\end{tabular}
\caption{Die Tabelle zeigt die Ergebnisse eines Profilers für die eigene Implementierung mit $\lambda = 10^{-5}$ und $\alpha = 0.5$.\\Gerechnet wurde auf einem Intel Xeon Gold 6130F@2.10GHz.}
\label{algorithm_analysis_profiler}
\end{table}
\setlength{\tabcolsep}{6pt}

Logischerweise erhöht sich mit der Anzahl an Iterationen auch die Laufzeit des Algorithmus. Aus Tabelle \ref{algorithm_analysis_table} lässt sich aber nicht direkt erkennen, ob sich auch die Laufzeit pro Iteration mit $\lambda$ verändert. In Abbildung \ref{run_time_per_iteration} beobachten wir im Mittel eine Zunahme der Laufzeit pro Iteration bei Senkung von $\lambda$. Des Weiteren steigt die Laufzeit, wenn mehr Gewicht auf eine Lasso-Bestrafung gelegt wird, also $\alpha$ nahe 1 gewählt wird. Auch hier lässt sich das Verhalten auf die erhöhte Anzahl von Null verschiedener Einträge zurückführen.

Da wir eine maximale Anzahl von $500$ Iterationen festgelegt haben, kommt es bei Werten $\lambda < 10^{-5}$ zu einer erhöhten Toleranz. Eine Erhöhung der Anzahl an Iterationen kann die Toleranz verringern, jedoch steigt damit auch die Laufzeit. Diese Obergrenze scheint für unsere Anwendung eine sinnvolle Wahl zu sein, muss aber für jeden Datensatz geeignet angepasst werden. Denkbar sind auch schwächere Konvergenzkriterien, die gegebenenfalls die Laufzeit verringern können.

\begin{figure}
\centering
\includegraphics[width = 0.9\textwidth]{figures/run_time_per_iteration.jpg}
\caption{In dieser Abbildung ist die Laufzeit pro Iteration bei Veränderung des Hyperparameters $\lambda$ auf einer logarithmischen Skala zu sehen. Da auch $\alpha$ in unseren Experimenten variiert worden ist und mehrere Sensoren betrachtet werden, sehen wir mehrere Punkte je $\lambda$. Im Mittel klar zu erkennen ist ein Anstieg der Laufzeit bei Verringerung der Stärke der Bestrafung $\lambda$.}
\label{run_time_per_iteration}
\end{figure}


%----------------------------------------------------------------------------------------
%	Experimentelle Überprüfung der berechneten Varinanz
%----------------------------------------------------------------------------------------


\subsection{Experimentelle Überprüfung der berechneten Varianzen}

In Abschnitt \ref{adjustment_of_variances} haben wir unterschiedliche Wege zur Berechnung der Hauptkomponenten und deren erklärte Varianz gezeigt. Um die Arbeit von Camacho et al. \cite{camacho} experimentell zu überprüfen, werden wir vier Kriterien definieren, welche auf den unterschiedlichen Vorgehensweisen basieren. Für jede dieser wird die Varianz der Residuen addiert und mit der Gesamtvarianz des Datensatzes normalisiert.
\begin{itemize}
\item TotQR: $\quad \frac{\sum_{j=1}^k R_{jj}^2 + \spur{\widehat{\mat E}^T\widehat{\mat E}}}{\spur{\mat X^T\mat X}} \quad$ (Vorgehensweies Zou et al.)
\item TotZB: $\quad \frac{\spur{\widehat{\mat B} \widehat{\mat Z}^T \widehat{\mat Z} \widehat{\mat B}^T} + \spur{\widehat{\mat E}^T\widehat{\mat E}}}{\spur{\mat X^T\mat X}} \quad$ (Vorgehensweise Camacho et al.)
\end{itemize}
Bezüglich der Notation haben wir uns an Abschnitt \ref{adjustment_of_variances} gehalten. Zwei weitere Kriterien TotQR* und TotZB* ergeben sich durch die Korrektur der Hauptkomponenten mit der Moore-Penrose-Inverse $\widehat{\mat Z}^* = \mat X \widehat{\mat B}^T (\widehat{\mat B}^T\widehat{\mat B})^+$. Falls alle Vorgehensweisen korrekt sind, können wir erwarten, dass jedes Kriterium den Wert $1$ hat. In Abbildung \ref{total_variance_validation} haben wir die Kriterien für unsere Experimente berechnet. Klar zu sehen ist, dass ohne eine Korrektur mit der Moore-Penrose-Inversen beide Varianten für die Varianzberechnung im Allgemeinem falsch sind. Auch wenn wir die Hauptkomponenten korrigieren, liefert die QR-Zerlegung keine richtigen Ergebnisse. Nur TotZB* hat in allen Fällen den Wert $1$ und ist damit der einzig korrekte Weg, Hauptkomponenten und Varianzen zu berechnen. Die von Zou et al. \cite{zou_sparsepca} vorgeschlagenen Varianten sind also falsch und sollten nicht verwendet werden. Somit können wir die Erkenntnisse aus \cite{camacho} experimentell bestätigen.

\begin{figure}
\centering
\includegraphics[width = 0.9\textwidth]{figures/total_variance_validation.png}
\caption{Zu sehen sind die Ergebnisse der unterschiedlichen Vorgehensweise bei der Berechnung der Hauptkomponenten und erklärter Varianzen. Nur die von Camacho et al. vorgeschlagene Variante TotZB* errechnet diese korrekt. Jeder Punkt entspricht eines unserer Experimente wie in Abschnitt \ref{application_frequency_data} beschrieben.}
\label{total_variance_validation}
\end{figure}
% Main chapter title
\chapter{Ausblick / Zusammenfassung}

% Chapter label
\label{conclusion}

\section{Einsetzbarkeit}
Wann ist die Methode sinnvoll einzusetzen?

Kritik 
results in very large residuals
highly correlated scores
q is no updated due to the normalization step

\section{Übertragbarkeit}
Übertragbarkeit auf andere Datensätze

\section{Ongoing Research / Weitere Techniken}

%----------------------------------------------------------------------------------------
%	BIBLIOGRAPHY
%----------------------------------------------------------------------------------------

\printbibliography[heading=bibintoc]


%----------------------------------------------------------------------------------------


\appendix



%----------------------------------------------------------------------------------------
%	Lineare Algebra
%----------------------------------------------------------------------------------------



\chapter{Lineare Algebra}
\label{linear_algebra}

Die Mathematik der Hauptkomponentenanalyse beruht im Wesentlichen auf Methoden der linearen Algebra. Aufgrund des Anwendungsfalls werden wir uns auf die Einführung der Grundbegriffe in reellen Vektorräumen beschränken.

\section{Orthogonalität}
\label{orthogonality}

\begin{defn}[Skalarprodukt \cite{jaenich}]
Sei $V$ ein reeller Vektorraum. Ein \textit{Skalarprodukt} in $V$ ist eine Abbildung $\inner{\cdot}{\cdot}: V \times V \longrightarrow \mathbb{R}$ mit den folgenden Eigenschaften:
\begin{enumerate}[(i)]
\item Für jedes $x \in V$ sind die Abbildungen
\begin{align*}
\inner{\cdot}{x}: V & \longrightarrow \mathbb{R} & \inner{x}{\cdot}: V & \longrightarrow \mathbb{R}\\
v & \longmapsto \inner{v}{x} & v & \longmapsto \inner{x}{v}
\end{align*}
linear. $\quad$ (Bilinearität)
\item $\inner{x}{y} = \inner{y}{x}$ für alle  $x,y \in V \quad$ (Symmetrie)
\item $\inner{x}{x} > 0$ für alle $x \neq 0 \quad$ (Positive Definitheit)
\end{enumerate}
\end{defn}

Allgemein versteht man unter einem \textit{euklidischem Vektorraum} ein Paar $(V, \inner{\cdot}{\cdot})$, welches aus einem reellem Vektorraum $V$ und einem Skalarprodukt $\inner{\cdot}{\cdot}$ auf $V$ besteht. Durch das Skalarprodukt wird eine Norm auf $V$ induziert
$$\norm{v} \defeq \sqrt{\inner{v}{v}}.$$
In den folgenden Kapiteln werden wir uns vor allem mit dem \textit{Standardskalarprodukt} im $\mathbb{R}^n$ beschäftigen. Dies ist gegeben durch 
$$\inner{x}{y} = x_1y_1 + \cdots + x_ny_n.$$

\begin{thm}[Verallgemeinerter Satz des Pythagoras \cite{anton}]
\label{pythagoras}
Für orthogonale Vektoren $u,v$ in einem euklidischem Vektorraum $V$ gilt
$$\norm{u+v}^2 = \norm u^2 + \norm v^2.$$
\end{thm}

\begin{defn}[Orthogonalität \cite{jaenich}]
Zwei Elemente $v, w$ eines euklidischen Vektorraumes $V$ heißen \textit{orthogonal} (geschrieben $v \perp w$) wenn ihr Skalarprodukt Null ist, d.h.
$$v \perp w \iff \inner{v}{w} = 0.$$
Eine Familie $(v_1, \ldots, v_n)$ in $V$ heißt \textit{orthogonal} oder \textit{Orthogonalsystem}, wenn
$$v_i \perp v_j \quad \text{für alle} \quad i \neq j.$$
Gilt zusätzlich $\inner{v_i}{v_i} = 1$ für alle $1 \leq i \leq n$, so spricht man von einem \textit{Orthonormalsystem}.
\end{defn}

%\begin{defn}[Orthonormalbasis \cite{fischer}]
%Sei $\inner{\cdot}{\cdot}: V \times V \longrightarrow \mathbb{R}$ ein Skalarprodukt. Ein System von Vektoren $(v_1, \ldots, v_n)$ in $V$ wird als \textit{Orthogonalbasis} (bzw. \textit{Orthonormalbasis}) bezeichnet, wenn folgende Bedingungen erfüllt sind:
%\begin{enumerate}[(i)]
%\item $(v_1, \ldots, v_n)$ ist eine Basis von $V$
%\item $(v_1, \ldots, v_n)$ ist ein Orthogonalsystem (bzw. Orthonormalsystem)
%\end{enumerate}
%\end{defn}

%\begin{thm}[Existenz einer Orthonormalbasis \cite{fischer}]
%Jeder endlichdimensionale euklidische Vektorraum besitzt eine Orthonormalbasis.
%\end{thm}

\begin{defn}[Orthogonale Matrix \cite{anton}]
Eine Matrix $\mat A \in \mathbb{R}^{n \times n}$ heißt 		\textit{orthogonal}, falls deren Zeilen- und Spaltenvektoren paarweise orthonormal bezüglich des Standardskalarprodukts sind, d.h.
$$\mat A^{\top} \mat A = \mathbb{1}_n.$$
\end{defn}

\begin{defn}[Orthogonalprojektion \cite{anton}]
Eine \textit{Orthogonalprojektion} auf einen Untervektorraum $U$ eines Vektorraumes $V$ ist eine lineare Abbildung $P_U \colon V \rightarrow V$, die für alle Vektoren $v \in V$ die beiden Eigenschaften
\begin{enumerate}[(i)]
\item $P_U(v) \in U \quad$ (Projektion)
\item $\langle P_U(v) - v , u \rangle = 0$ für alle $u \in U \quad$ (Orthogonalität)
\end{enumerate}
erfüllt.
\end{defn}

Mithilfe einer Orthogonalbasis für $U$ lässt sich aus dieser Definition eine explizite Lösung für die Orthogonalprojektion $P_U(v)$ herleiten.

\begin{thm}[\cite{anton}]
\label{orthogonal_projection_theorem}
Ist $(u_1, \ldots, u_r)$ eine Orthogonalbasis von $U$, so gilt für alle $v \in V$
$$P_{U}(v) = \sum_{i=1}^r \frac{\langle v, u_i \rangle}{\langle u_i, u_i \rangle} u_i.$$
\end{thm}

Später werden wir die Orthogonalprojektion in einer anderen Form nutzen. Wir können die Projektion als Matrix-Vektor-Produkt auffassen. Verwenden wir das Standardskalarprodukt gilt mit einer Orthogonalbasis $(u_1, \ldots, u_r)$ von $U$
\begin{align}
P_U(v) = \sum_{i=1}^r \frac{v^{\top} u_i}{u_i^{\top} u_i} u_i = \sum_{i=1}^r \frac{u_i u_i^{\top}}{u_i^{\top} u_i}v = \mat A \mat A^{\top} v,
\end{align}
wobei $\mat A = \begin{bmatrix} \frac{u_1}{\norm{u_1}} & \cdots & \frac{u_r}{\norm{u_r}} \end{bmatrix} \in \mathbb{R}^{n \times r}$. Die Orthogonalitätsbedingung in Theorem \ref{orthogonal_projection_theorem} kann auch weggelassen werden. Ist $(u_{1},\ldots ,u_{r})$ eine beliebige Basis von $U$, so gilt
\begin{align}
P_U(v) = \mat A (\mat A^\top \mat A)^{-1} \mat A^\top v.
\end{align}
Wir nennen $\mat P = \mat A (\mat A^\top \mat A)^{-1} \mat A^\top$ die \textit{orthogonale Projektionsmatrix}. Mithilfe von Theorem \ref{pythagoras} lässt sich zeigen, dass der orthogonal auf den Unterraum projizierte Vektor den Abstand zwischen dem Ausgangsvektor und dem Unterraum minimiert.

\begin{thm}[\cite{anton}]
Sei $U$ ein Unterraum eines euklidischen Vektorraumes $V$. Dann ist $P_U(v)$ die beste Näherung von $u$ in $U$, d.h.
$$\norm{P_U(v) - v}^2 \leq \norm{u - v}^2 \quad \text{für alle } u \in U$$
\end{thm}

\section{Matrixzerlegungen}
\label{matrix_decomposition}

In diesem Abschnitt führen wir zwei wichtige Matrixzerlegungen ein, die auch in vielen Bereichen der Numerik Anwendung finden.

\begin{defn}[Eigenwert, Eigenvektor \cite{anton}]
Sei $\mat{A} \in \rnn$. Ein von Null verschiedener Vektor $x \in \rn$ heißt \textit{Eigenvektor} von $\mat{A}$, falls
$$\mat{A}x = \lambda x$$
für einen Skalar $\lambda \in \mathbb{R}$. Die Zahl $\lambda$ heißt \textit{Eigenwert} von $\mat{A}$.
\end{defn}

\begin{defn}[Diagonalisierbarkeit \cite{anton}]
Eine quadratische Matrix $\mat A \in \rnn$ heißt \textit{diagonalisierbar}, falls eine invertierbare Matrix $\mat V$ existiert, so dass $\mat{\Lambda} = \mat{V}^{-1}\mat{A}\mat{V}$ Diagonalgestalt hat.
\end{defn}

Es gibt verschiedene Kriterien für die Diagonalisierbarkeit von Matrizen. Für unsere spätere Anwendung interessieren wir uns vor allem für die Frage, ob es zu einer gegebenen Matrix $\mat{A} \in \rnn$ eine orthogonale Matrix $\mat{V}$ gibt, die $\mat{A}$ diagonalisiert. Eine derartige Diagonalisierung wird auch als \textit{Hauptachsentransformation} bezeichnet. Dieser Name stammt ursprünglich aus der Theorie der Kegelschnitte. Hierbei ist eine Hauptachsentransformation eine orthogonale Abbildung, welche die Koordinatenachsen in die Richtungen der beiden \textit{Hauptachsen} überführt. Wir wollen uns aber vorerst nicht mit dieser geometrischen Interpretation beschäftigen, sondern mit einem mathematisch äquivalenten, in den Anwendungen aber wichtigeren Problem.

\begin{thm}[Hauptachsentransformation \cite{jaenich}]
Sei $\mat{A} \in \rnn$ eine symmetrische Matrix. Dann gibt es eine orthogonale Transformation $\mat{V}$, welche $\mat{A}$ in eine Diagonalmatrix $\mat{\Lambda} \defeq \mat{V}^{-1}\mat{A}\mat{V}$ der Gestalt
$$\mat{\Lambda} = \begin{bmatrix}
    \lambda_{1} & & & & & & \\
    & \ddots & & & & & \\
    & & \lambda_1 & & & & \\
    & & & \ddots & & & \\
    & & & &\lambda_r & & \\
    & & & & & \ddots & \\
    & & & & & & \lambda_{r}
  \end{bmatrix}$$
überführt. Hierbei sind $\lambda_1, \ldots, \lambda_r$ die verschiedenen Eigenwerte von $\mat{A}$.
\end{thm}

Zusammenfassend besitzt eine symmetrische Matrix also immer eine Zerlegung $\mat A = \mat V \mat \Lambda \mat V^{\top}$. Man kann $\mat{V}$ konstruieren, so dass die Spalten genau den Eigenvektoren von $\mat{A}$ entsprechen. Wir werden diese Umformung in späteren Kapiteln unter dem Begriff \textit{Eigenwertzerlegung} (Englisch: \textit{Eigenvalue Decomposition}) verwenden. 

Eine eng verwandte, aber vielseitigere Faktorisierung von Matrizen ist die \textit{Singulärwertzerlegung}. Sie ermöglicht eine allgemeine Zerlegung auch von nicht quadratischen oder nicht symmetrischen Matrizen.

\begin{thm}[Singulärwertzerlegung \cite{schaback}]
Jede Matrix $\mat{A} \in \rmn$ besitzt eine \textit{Singulärwertzerlegung} 
$$\mat{A} = \mat{U}\mat{D}\mat{V}^{\top}$$
mit orthogonalen Matrizen $\mat U \in \mathbb{R}^{m \times m}$ und $\mat V \in \rnn$, sowie der Diagonalmatrix $\mat{D} = (\sigma_j\delta_{ij}) \in \rmn$.
\end{thm}

\begin{defn}[Singulärwert]
Die positiven Diagonaleinträge $\sigma_{i} > 0$ von $\mat D$ werden \textit{Singulärwerte} genannt.
\end{defn}

Singulärwerte einer Matrix $\mat A$ sind eindeutig bestimmt und stehen durch $\sigma_i = \sqrt{\lambda_i}$ in einer engen Beziehung mit den Eigenwerten $\lambda_i$ von $\mat A^T\mat A$. Konventionell werden die Singulärwerte von $\mat D$ absteigend sortiert, d.h. $\sigma _{1} \geq \cdots \geq \sigma _{r}$. Geometrisch bedeutet diese Zerlegung, dass sich die Matrix $\mat A$ in zwei Drehungen $\mat U, \mat V$ und eine Streckung unterteilen lässt. Dabei korrespondieren die Streckungsfaktoren mit den Einträgen der Diagonalmatrix $\mat D$.


\section{Matrix Approximation}
\label{matrix_approximation}

In diesem Abschnitt werden wir zwei Approximationsprobleme für Matrizen formulieren, die eine explizite Lösung besitzen. Zunächst führen wir dafür eine Matrixnorm ein, von welcher wir auch später sehr häufig Gebrauch machen werden.

\begin{defn}[Frobeniusnorm \cite{schaback}]
Für eine Matrix $\mat A \in \rmn$ ist die \textit{Frobeniusnorm} definiert durch
$$\norm{\mat A}_F = \left( \sum_{i=1}^{m} \sum_{j=1}^{n} \lvert a_{ij} \rvert ^{2} \right) ^{\frac{1}{2}}.$$
\end{defn}

Man zeigt leicht, dass $\norm{\mat A}_F^2 = \spur{\mat A^{\top} \mat A}$ gilt.
Eine weitere wichtige Eigenschaft von Matrizen ist der \textit{Rang}.

\begin{defn}[Rang \cite{anton}]
Die Dimension des Zeilen- und des Spaltenraumes einer Matrix $\mat A$ heißt \textit{Rang} von $\mat A$ und wird mit $\rang{\mat A}$ bezeichnet.
\end{defn}

Wir möchten nun eine Matrix $\mat A$ durch eine andere, simplere Matrix $\widehat{\mat A}$ mit niedrigerem Rang approximieren. Dieses Problem fällt unter die Kategorie \textit{low rank approximation}, welche eine enge Verbindung zur Hauptkomponentenanalyse aufweist. Mithilfe der Singulärwertzerlegung können wir eine explizite Lösung angeben.

\begin{thm}[Eckart-Young-Mirsky-Theorem \cite{eckart}]
Sei $\mat A \in \rmn$ mit $m \leq n$ und 
$$\mat{A} = \mat{U}\mat{D} \mat{V}^{\top}$$
eine Singulärwertzerlegung von $\mat{A}$. Wir partitionieren $\mat{U}, \mat{D}$ und $\mat{V}$ wie folgt:
$$\mat{U} =: \begin{bmatrix} \mat{U}_1 & \mat{U}_2\end{bmatrix}, \quad 
\mat{D} =: \begin{bmatrix} \mat{D}_1 & 0 \\ 0 & \mat{D}_2 \end{bmatrix},\quad \mat{V} =: \begin{bmatrix} \mat{V}_1 & \mat{V}_2 \end{bmatrix},$$
wobei $\mat{U}_{1} \in \mathbb{R}^{m\times r}$, $\mat{D} _{1} \in \mathbb{R}^{r\times r}$ und $\mat{V}_{1} \in \mathbb{R}^{n\times r}$. Dann löst die abgeschnittene Singulärwertzerlegung (Englisch: \textit{truncated singular value decomposition)}
$$\widehat{\mat{A}}^* = \mat{U}_1 \mat{D}_1 \mat{V}_1^{\top},$$
das Approximationsproblem
\begin{align}
\min_{\operatorname{rank}(\widehat{\mat{A}}) \leq r} \|\mat{A}-\widehat{\mat{A}}\|_{\text{F}} = \|\mat{A}-\widehat{\mat{A}}^*\|_{\text{F}} = \sqrt{\sigma^2_{r+1} + \cdots + \sigma^2_m},
\end{align}
wobei $\sigma_i$ die Singulärwerte von $\mat A$ sind. Der Minimierer $\widehat{\mat{A}}^*$ ist genau dann eindeutig, wenn $\sigma_{r+1} \neq \sigma_{r}$.
\end{thm}

Das Eckart-Young-Mirsky-Theorem wird es uns in Abschnitt \ref{pca_theorems} ermöglichen, eine wertvolle Eigenschaft der Hauptkomponentenanalyse zu zeigen. In Anwendung korrespondieren die Singulärwerte mit dem Rekonstruktionsfehler und die Rang-Bedingung an $\widehat{\mat A}$ mit der Komplexität des Modells.

Ein anderes Approximationsproblem für Matrizen ist das \textit{orthogonale Procrustes Rotationsproblem}. Hierbei sind uns zwei Matrizen $\mat M$ und $\mat N$ gegeben, welche durch eine orthogonale Transformation ineinander überführt werden sollen. Wieder hilft uns die Singulärwertzerlegung bei der Findung einer Lösung.

\begin{thm}[Procrustes Rotationsproblem \cite{gower}]
Seien $\mat M \in \mathbb{R}^{n \times p}$, $\mat N \in \mathbb{R}^{n \times k}$ und $\mat M^{\top} \mat N = \mat{U}\mat{D} \mat{V}^{\top}$ eine Singulärwertzerlegung. Dann löst
$$\widehat{\mat A} = \mat U \mat V^{\top}$$
das Approximationsproblem
\begin{align}
\label{procrustes_rotation}
\begin{gathered}
\widehat{\mat A} = \argmin_{\mat A} \norm{\mat M - \mat N \mat A^{\top}}_F^2\\
\text{unter der Nebenbedingung, dass } \mat A^{\top} \mat A = \mat{\mathbb{1}}_{k \times k}.
\end{gathered}
\end{align}
\end{thm}

In Abschnitt \ref{spca_numerical_solution} wird sich (\ref{procrustes_rotation}) als Subproblem der dünnbesetzten Hauptkomponentenanalyse herausstellen.

\section{Pseudoinverse}

In vielen Anwendungen der numerischen Mathematik benötigt man für die Angabe einer expliziten Lösung die Inverse einer Matrix. Allerdings sind diese nur für quadratische, nichtsinguläre Matrizen definiert. Daher ist eine Verallgemeinerung des Konzepts nötig. Verallgemeinerte Inversen werden in der Literatur nicht einheitlich gehandhabt und orientieren sich häufig an der zu lösenden Aufgabenstellung. Für unseren Anwendungsfall in Kapitel \ref{sparse_pca} benutzen wir die \textit{Moore-Penrose-Inverse}. Wir werden die beiden Begriffe Pseudoinverse und Moore-Penrose-Inverse daher synonym verwenden.

\begin{defn}[Moore-Penrose-Inverse \cite{israel}]
Die \textit{Moore-Penrose-Inverse} einer Matrix $\mat A \in \mathbb{R}^{m \times n}$ ist die eindeutig bestimmte Matrix $\mat{A}^+ \in \mathbb{R}^{n \times m}$, welche die folgenden vier Eigenschaften erfüllt
\begin{enumerate}[(i)]
\item $\mat A \mat A^+ \mat A = \mat A$
\item $\mat A^+ \mat A \mat A^+ = \mat A^+$
\item $(\mat A \mat A^+)^T = \mat A \mat A^+$
\item $(\mat A^+ \mat A)^T = \mat A^+ \mat A$
\end{enumerate}
\end{defn}

Für quadratische, nichtsinguläre Matrizen entspricht die Pseudoinverse der regulären Inversen $\mat A^+ = \mat A^{-1}$. Sind die Spalten bzw. Zeilen der Matrix $\mat A$ linear unabhängig, gilt $\mat A^{+} = (\mat A^T \mat A)^{-1} \mat A^T$ bzw. $\mat A^{+} = \mat A^T (\mat A \mat A^T)^{-1}$. Im Allgemeinen kann die Pseudoinverse mithilfe einer Singulärwertzerlegung berechnet werden. Ist $\mat A = \mat U \mat D \mat V^T$ eine Singulärwertzerlegung von $\mat A$, gilt $\mat A^+ = \mat V \mat D^+ \mat U^T$. Für eine Diagonalmatrix $\mat D$ entsteht die Pseudoinverse durch Transponieren und Invertieren der von Null verschiedenen Elemente
$$(\mat D)_{ij}^{+} = \begin{cases} \frac{1}{\sigma_{i}} \quad \text{falls } i = j \wedge \sigma _{i} \neq 0\\ 0 \quad \text{ sonst}\end{cases}.$$


\end{document}  
