
%----------------------------------------------------------------------------------------
%	PACKAGES AND OTHER DOCUMENT CONFIGURATIONS
%----------------------------------------------------------------------------------------

\documentclass{beamer}

\usetheme[hideothersubsections]{Goettingen}
\usepackage{german}
\usepackage{algorithm}
\usepackage{algpseudocode}

\newcommand{\mat}[1]{\mathbf{#1}}
\DeclareMathOperator*{\argmax}{arg\,max}
\DeclareMathOperator*{\argmin}{arg\,min}
\newcommand{\norm}[1]{\left\lVert #1 \right\rVert}

%----------------------------------------------------------------------------------------
%	BEGIN DOCUMENT
%----------------------------------------------------------------------------------------


\begin{document}

\title{Sparse Principal Component Analysis}
\subtitle{for Frequency Data}   
\institute{Institute for Numerical Simulation}
\author{Tobias Bork} 
\date{\today}
\titlegraphic{	\includegraphics[scale=0.04]{figures/university_of_bonn.png} 				\hspace{0.5cm}
				\includegraphics[scale=0.10]{figures/fraunhofer_scai.png}}

\begin{frame}
\titlepage
\end{frame}

%\begin{frame}
%\frametitle{Table of Contents}
%\tableofcontents
%\end{frame}

%----------------------------------------------------------------------------------------
%	Introduction
%----------------------------------------------------------------------------------------

\section{Introduction} 
\begin{frame}
\frametitle{Titel} 
Die einzelnen Frames sollte einen Titel haben 
\end{frame}

%----------------------------------------------------------------------------------------
%	Principal Component Analysis
%----------------------------------------------------------------------------------------

\section{PCA} 
\subsection{Idea}
\begin{frame}\frametitle{Aufz\"ahlung}
\begin{itemize}
\item Einf\"uhrungskurs in \LaTeX  
\item Kurs 2  
\item Seminararbeiten und Pr\"asentationen mit \LaTeX 
\item Die Beamerclass 
\end{itemize} 
\end{frame}

\subsection{Mathematical Formulations}
\subsection{Theorems}
\subsection{Limits of Usability}
\subsection{Application}

%----------------------------------------------------------------------------------------
%	Fundamentals
%----------------------------------------------------------------------------------------

\section{Fundamentals}

\subsection{Regression}
\subsection{Sparsity inducing Norms}

%----------------------------------------------------------------------------------------
%	Sparse Principal Component Analysis
%----------------------------------------------------------------------------------------

\section{Sparse PCA}

\subsection{Mathematical Formulation}
\subsection{Numerical Solution}

\begin{frame}
\begin{algorithm}[H]
  \scriptsize
    \caption{General SPCA Algorithm}
    \begin{algorithmic}[1]
        \Procedure{SPCA}{$A,B$}
        	\State $\mat A \gets \mat V[,1 \colon k]$, the loadings of the first k ordinary principal components
            \While{not converged} \Comment{Definiere Abbruchkriterium}
                \State Given a fixed $\mat A = [\alpha_1, \ldots, \alpha_k]$, solve the elastic net problem $$\beta_j = \argmin_{\beta} \norm{\mat X \alpha_j - \mat X \beta}^{2} + \lambda \norm{\beta}^2 + \lambda_{1,j}\norm{\beta}_{1}$$
                \State For a fixed $\mat B = [\beta_1, \ldots, \beta_k]$, compute the SVD of $$\mat X^T \mat X \mat B = \mat U \mat D \mat V^T$$
                \State $\mat A \gets \mat U \mat V^T$
            \EndWhile
            \State $\hat{V}_j = \frac{\beta_j}{\norm{\beta_j}}$ for $j = 1, \ldots, k$
        \EndProcedure
    \end{algorithmic}
\end{algorithm} 
\end{frame}

\subsection{Adjusted Variances}
\subsection{p >> n case}

%----------------------------------------------------------------------------------------
%	Application
%----------------------------------------------------------------------------------------

\section{Application}

\subsection{Dataset}
\subsection{Results}

%----------------------------------------------------------------------------------------
%	References
%----------------------------------------------------------------------------------------

\section{References}
\begin{frame}

\frametitle{References}

\begin{thebibliography}{9}
\bibitem[Beamerpaket]{paket} \emph{Beamer Paket} \\ 
\text{http://latex-beamer.sourceforge.net/}
\bibitem[Beamerdokumentation]{doku} \emph{User's Guide to the Beamer} 
\bibitem[Dante]{dante} \emph{DANTE e.V.} \text{http://www.dante.de}   
\end{thebibliography}


\end{frame}



\end{document}
