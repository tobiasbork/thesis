% Main chapter title
\chapter{Hauptkomponentenanalyse}

% Chapter label
\label{pca}

\section{Motivation}

Die Hauptkomponentenanalyse ist ein weitverbreites multivariates statistisches Verfahren zur Dimensionsreduktion. Multivariate Verfahren zielen darauf ab, die in einem Datensatz enthaltene Zahl der Variablen zu verringern, ohne die darin enthaltene Information wesentlich zu reduzieren. Dadurch können umfangreiche Datensätze strukturiert, veranschaulicht und vereinfacht werden. Als Teil der explorativen Statistik ...

Somit findet die Hauptkomponentenanalyse in vielen Bereichen Anwendung.
Ein paar Beispiele (hand written zip code classification or human face recognition).

Das dahinterstehende mathematische Problem kann auf mehrere Weisen beschrieben werden. Zunächst wollen wir die Hauptkomponentenanalyse so konstruieren, dass die Idee des minimalen Informationsverlust im Vordergrund steht. Anschließend werden wir das Problem auf ein Singulärwertzerlegung zurückführen, die auch zur effizienten Implementierung genutzt wird. Des Weiteren werden wir die Hauptkomponentenanalyse als Regressionsproblem betrachten, und die geometrische Interpretation weiter verdeutlichen. Zu Schluss werden wir die Äquivalenz dieser Formulierungen und einige theoretische Aussagen zeigen.

\section{Einführung}

Die zentrale Idee der Hauptkomponentenanalyse besteht darin, die bestehenden Variablen in neue, unkorrelierte Variablen zu überführen, ohne dabei Information zu verlieren. Als Maß für den Informationsgehalt der Daten wird hierbei die Varianz verwendet. Konkret konstruieren wir Variablen, die sich aus Linerakombinationen der Alten zusammensetzen. Dabei sollen die neuen Variablen der Wichtigkeit nach sortiert sein. In anderen Worten enthält die erste Variable die meiste Information bzw. die größte Varianz, dann die zweite, usw.
Die eigentliche Dimensionsreduktion findet dann durch Selektierung statt. Je nach Komplexität des Modells und Informationsverlust können so mehr oder weniger ausgewählt werden. Somit haben wir eine kleine, neue Menge an Variablen, die aber trotzdem den Großteil an Informationen / Varianz beinhaltet.




Wie müssen die Daten aufbereitet sein? Zentriert und skaliert? Erklären verschiedener Methoden und deren Auswirkungen für die Daten. Korrelationsmatrix oder Kovarianzmatrix?




\section{Konstruktion}

\section{Problemformulierung}
\subsection{Formulierung als Singulärwertzerlegung}
\subsection{Formulierung als Regressionsproblem}
\subsection{Formulierung als beste lineare Mannigfaltigkeit Approximation}
Ausgehend von diesem Regressionsproblem werden wir im nächsten Kapitel die Variante der dünnbesetzten Hauptkomponentenanalyse beschreiben.

\section{Theoretische Aussagen}