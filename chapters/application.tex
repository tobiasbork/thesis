% Main chapter title
\chapter{Anwendung}

% Chapter Label
\label{application}

In diesem Kapitel beschäftigen wir uns mit der Anwendung der dünnbesetzten Hauptkomponentenanalyse auf Frequenzdaten einer Maschine.
Nicht unbedingt an der niedrigdimensionalen Repräsentation interessiert, sondern viel mehr an ...
 
Kurze Beschreibung der Idee, was wir uns erhoffen?
Zunächst werden wir in Abschnitt \ref{data_set} den Datensatz beschreiben und Vorverarbeitungsschritte erläutern. 

Ob eine derartige Methode für diesen Datensatz sinnvoll war, werden wir in Kapitel \ref{conclusion} diskutieren.


%----------------------------------------------------------------------------------------
%	Beschreibungs des Datensatzes
%----------------------------------------------------------------------------------------


\section{Beschreibung des Datensatzes}
\label{data_set}

grobe Einführung Messdaten akkustischer und beschleuninungsensoren, welche die Vibration einer Maschine. (DARF ich hier die Idee nennen, warum die Messungen erhoben worden sind? Wir erhoffen bestimmte Frequenzen rauszufiltern, welche eine Aussage über die Partikelgröße ermöglichen.
Was wurde geändert zwischen den Messungen? Rotation speed, flow pressure, und ...
Sensoren an verschiedenen Teilen der Maschine angebracht, z.B. an der Mahlkammer, Auslassrohr oder am Sichter..
Messungen mit und ohne Material, drei Messungen doppelt, 8 verschiedene Betriebszustände, einmal mit einmal ohne Material 
 Maschinenfrequenzen rausfiltern.

Dadurch, dass die Messungen mit einer hohen Abtastrate durchgeführt worden sind, haben wir es mit einem hochdimensionalen Datensatz zu tun. Für jeden an der Maschine angebrachten Sensor haben wir somit $p = 5120000$ Variablen. Mit nur $n = 33$ Messungen, wobei Teile mehrmals aufgenommen worden sind, sind wir mit einer \textit{high dimension low sample size (HDLSS)} Situation konfrontiert. Auf dieser Art Datensatz sind viele Methoden des unüberwachten Lernens unbrauchbar, weshalb eine Dimensionsreduktion sinnvoll sein kann. In diesem Fall sind wir aber weniger an der niedrigdimensionalen Repräsentation der Daten interessiert als an der Herausfilterung wichtiger Frequenzen, die eine Trennung der verschiedenen Messungen ermöglicht. Diese Möglichkeit der Interpretation war Anlass für die Verwendung der dünnbesetzten Hauptkomponentenanalyse. Des Weiteren haben wir in Abschnitt \ref{pca_theorems} theoretische Aussagen für eine Inkonsistenz der Hauptachsen in einer solchen Situation präsentiert. Mit der dünnbesetzten Variante können wir die Konsistenz wiederherstellen.

\section{Vorverarbeitung der Daten}

Vor der Anwendung der dünnbesetzte Hauptkomponentenanalyse auf den Datensatz haben wir einige Vorverarbeitungsschritte vorgenommen. Anfänglich sind uns Zeitreihen für sowohl die Mikrofone als auch für die Beschleunigungssensoren gegeben. Da die Messungen zu zufälligen Zeitpunkten bei laufendem Mahlprozess gestartet worden sind, können einzelne Zeitpunkte nicht direkt miteinander verglichen werden. Somit haben wir eine Fouriertransformation der Daten angewandt. Hierdurch arbeiten wir anstatt auf der Zeitreihe also mit Frequenzdaten. In Abbildung REF zeigen wir beispielhaft das Ergebnis einer Fouriertransformation. Kurze Erklärung was wir auf x-y-Achse sehen. Dadurch erhoffen wir uns zusätzlich, dass Rauscheffekte besser erkennbar sind.

Es wird sich zeigen, dass der Algorithmus für die dünnbesetzte Hauptkomponentenanalyse sehr rechenintensiv sein kann. Daher haben wir uns entschieden nur einen Teil der ursprünglichen Zeitreihe zu verwenden. In Abbildung sieht man, dass sich die Frequenzen über die Zeit kaum ändern, was daran liegt, dass konstant Material in die Maschine zugeführt wird. Somit können wir die Dimension um einen Faktor $100$ reduzieren ohne wichtige Informationen zu verlieren. Des Weiteren  wurden Teile der Frequenzen, welche außerhalb des Frequenzbereichs des jeweiligen Sensors liegen abgeschnitten. 

Ähnlich wie bei der klassischen Hauptkomponentenanalyse werden die Daten vor Anwendung zentriert, so dass die Varianz zwischen den Variablen vergleichbar ist. Eine Standardisierung haben wir nicht vorgenommen. Standardisieren ja/nein? 



%----------------------------------------------------------------------------------------
%	Anwedung auf Frequenzdaten
%----------------------------------------------------------------------------------------


\section{Anwendung auf Frequenzdaten}
\label{application_frequency_data}

Unsere Implementierung ermöglicht es verschiedene Modellparameter zu wählen. Für eine Beschränkung der Laufzeit setzen wir eine maximale Anzahl an Iterationen von $500$ und eine Toleranz von $10^{-4}$. Falls nach $500$ Iterationen die vorgegebene Toleranz noch immer nicht erreicht ist, werden wir dies im Folgenden kenntlich machen. Ein weiterer Parameter den es zu wählen gilt, ist die Anzahl zu berechnender Hauptkomponenten. Wie bereits in \ref{sparse_pca_theorems}beschrieben kann sich die Zusammensetzung der ersten Hauptachsen durchaus ändern, falls man weitere Hauptkomponenten berechnen möchte. In unserem Fall haben wir daher einen Durchlauf mit $2$ bzw. $10$ Hauptkomponenten gestartet. Interessanter jedoch ist die Wahl der Hyperparameter $\lambda_1$ und $lambda_2$, welche wesentlichen Einfluss auf die Ergebnisse besitzen. 

Mithilfe eines BIC und AIC Kriteriums auswählen?

Exemplarisch werden wir uns nun mit einem der akustischen und einem der Vibrationssensoren beschäftigen. An dieser Stelle möchten wir erwähnen, dass wir die Sensoren getrennt betrachten, d.h. die dünnbesetzte Hauptkomponentenanalyse einzeln auf die Sensoren anwenden.
(Sonst Standardisierung aufgrund verschiedener Skalen notwendig, welche zu schlechteren Ergebnissen führt)


%----------------------------------------------------------------------------------------
%	Auswertung der Ergebnisse
%----------------------------------------------------------------------------------------


\section{Auswertung der Ergebnisse}
\label{evaluation}

Vergleich Tabelle PCA / Sparse PCA (Loadings)

Exemplarisch zeigen, dass wir dünnbesetzte Hauptachsen haben. Klare Trennung z.B. nach rotation speed. Erklärte Varianz dropt.

Um einen Vergleich zu ermöglichen, haben wir zeitgleich die klassische Hauptkomponentenanalyse auf den Datensatz angewandt.
Wir möchten nun ausgewählte Ergebnisse vorstellen. Zeitgleich werden wir die Resultate mit 



Effekte: Recomputation necessary if dimension changing



%----------------------------------------------------------------------------------------
%	Vergleich zur klassischen Hauptkomponentenanalyse
%----------------------------------------------------------------------------------------


\section{Vergleich zur klassischen Hauptkomponentenanalyse}
\label{comparison}


%----------------------------------------------------------------------------------------
%	Effekte der Hyperparamter
%----------------------------------------------------------------------------------------


\section{Effekte der Hyperparameter}
\label{parameter_effects}
Veränderung des Hyperparameters und dessen Effekte
\subsection{Zeit}
evtl. Zeit pro Iteration betrachten?
\subsection{Anzahl an Iterationen}
\subsection{Effekt auf Resultate}