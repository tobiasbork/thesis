% Main chapter title
\chapter{Dünnbesetzte Hauptkomponentenanalyse}

% Chapter label
\label{sparse_pca}

Ein Nachteil der Hauptkomponentenanalyse ist, dass sich die neuen Variablen meist aus einer Linearkombination aller bestehenden Variablen zusammensetzt. Dies macht es besonders für hochdimensionale Daten schwierig die Hauptachsen zu interpretieren. Oft können somit nicht die relevanten features/Variablen herausgelesen werden. Es kann durchaus passieren, dass nicht alle Variablen relevant zur Strukturerkennung sind. 

\section{Motivation}

\section{Problemformulierung}
NP-schwere Formulierung

\section{Relaxation / Approximation Ideen}

\section{Konstruktion}

Sparse PCA Kriterium.

$$(\hat{\mat{A}}\hat{\mat{B}}) = \argmin_{\mat{A}, \mat{B}} \sum_{i=1}^{n} \norm{x_i - \mat{A}\mat{B}^Tx_i}^2 + \lambda \sum_{j=1}^{k}\norm{\beta_j}^2 + \sum_{j=1}^k \lambda_{1,j} \norm{\beta_j}_1$$

subject to $\mat{A}^T\mat{A} = I_{k \times k}$
oblique projections AB

\section{Theoretische Aussagen Sparse PCA}
z.B. wie werden neue Varianzen berechnet