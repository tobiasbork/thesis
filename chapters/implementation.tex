% Main chapter title
\chapter{Implementierung}

% Chapter label
\label{implementation}

In diesem Kapitel werden wir einen Algorithmus beschreiben, der das Sparse PCA Kriterium (\ref{spca_criterion}) minimiert. Dabei sehen wir uns mit einem nicht-konvexem Optimierungsproblem konfrontiert. Allerdings können wir ausnutzen, dass (\ref{spca_criterion}) konvex bezüglich $\widehat{\mat A}$ bzw. $\widehat{\mat B}$ ist, falls wir eine der beiden Matrizen fixieren. Im Folgenden werden wir zunächst eine allgemeine numerische Lösung präsentieren und die Komplexität des assoziierten Algorithmus bestimmen. In einer HDLSS-Situation werden wir diesen in \ref{numerical_solution_p_greater_n} leicht abändern, um eine effiziente Berechnung zu garantieren. Zu Schluss dieses Kapitels diskutieren wir Details einer eigenen Implementierung in Python.


%----------------------------------------------------------------------------------------
%	Numerische Lösung
%----------------------------------------------------------------------------------------


\section{Numerische Lösung}
\label{spca_numerical_solution}

Für eine einfache Übersicht werden wir das Sparse PCA Kriterium hier wiederholen.
\begin{align}
\label{spca_criterion_restatement}
\begin{split}
(\widehat{\mat{A}}, \widehat{\mat{B}}) = \argmin_{\mat{A}, \mat{B}} \sum_{i=1}^{n} \norm{x_i - \mat{A}\mat{B}^Tx_i}_2^2 + \lambda_2 \sum_{j=1}^{k}\norm{\beta_j}_2^2 + \sum_{j=1}^k \lambda_{1,j} \norm{\beta_j}_1\\
\text{unter der Nebenbedingung, dass } \mat{A}^T\mat{A} = I_{k \times k}.
\end{split}
\end{align}
Wie beschrieben handelt es sich bei (\ref{spca_criterion_restatement}) um ein bikonvexes Optimierungsproblem. Daher liegt es nahe einen alternierenden Ansatz zu wählen, um das Problem numerisch zu lösen. Somit fixieren wir im Folgenden eine der beiden Matrizen.

\subsubsection{$\mat B$ gegeben $\mat A$:}

Wir wenden uns zunächst der Verlustfunktion zu. Hierfür sei $\mat A_{\perp} \in \mathbb{R}^{p \times (p-k)}$ eine orthonormale Matrix, so dass $[\mat A ; \mat A_{\perp}]$ $p \times p$ orthonormal ist. Dann gilt

\begin{align*}
\sum_{i=1}^{n} \norm{x_i - \mat{A}\mat{B}^Tx_i}_2^2 & = \norm{\mat X - \mat X \mat B \mat A^T}_F^{2}\\
& = \norm{\mat X \mat A_{\perp}}_F^2 + \norm{\mat X \mat A - \mat X \mat B}_F^2\\
& = \norm{\mat X \mat A_{\perp}}_F^2 + \sum_{j=1}^{k}\norm{\mat X \alpha_j - \mat X\beta_j}_2^2.
\end{align*}
Somit reduziert sich (\ref{spca_criterion_restatement}) für fixes $\mat A$ auf das Lösen von $k$ Elastic Net Problemen
\begin{align}
\label{sub_problem_enet}
\widehat{\beta}_j = \argmin_{\beta_j} \norm{Y_j^* - \mat X \beta_j}_2^2 + \lambda_2 \norm{\beta_j}_2^2 + \lambda_{1,j}\norm{\beta_j}_1,
\end{align}
wobei $Y_j^* = \mat X \alpha_j$ für alle $1 \leq j \leq k$. In Kapitel \ref{fundamentals} haben wir uns ausführlich mit Elastic Nets beschäftigt und ein effizientes Koordinaten-Abstiegsverfahren zur Lösung dieser präsentiert.

\subsubsection{$\mat A$ gegeben $\mat B$:}

Fixieren wir die Matrix $\mat B$, können wir uns auf das Minimieren der Verlustfunktion $\norm{\mat X - \mat X \mat B \mat A^T}_F^{2}$ beschränken, da die Bedingungen an $\beta_j$ nicht von Relevanz beim Optimieren über $\mat A$ sind. Somit reduziert sich (\ref{spca_criterion_restatement}) für fixes $\mat B$ auf das Lösen von
\begin{align}
\label{sub_problem_procrustes}
\begin{split}
\widehat{\mat A} = \argmin_{\mat A}\norm{\mat X - \mat X \mat B \mat A^T}_F^2\\
\text{unter der Nebenbedingung, dass } \mat A \mat A^T = \mat I_{k \times k}.
\end{split}
\end{align}
Für dieses Optimierungsproblem lässt sich eine explizite Lösung angeben. Es ist eine Form von Procrustes Rotationsproblem, welches wir ebenfalls in Kapitel \ref{fundamentals} beschrieben haben. Die Matrix $\mat X \mat B$ soll durch Multiplikation mit einer orthogonalen Matrix $\mat A$ in $\mat X$ überführt werden. Berechnen wir eine Singulärwertzerlegung von $(\mat X^T \mat X) \mat B = \mat U \mat D \mat V^T,$ ist die Lösung für (\ref{sub_problem_procrustes}) gegeben durch
\begin{align}
\widehat{\mat A} = \mat U \mat V^T.
\end{align}


%----------------------------------------------------------------------------------------
%	Algorithmus
%----------------------------------------------------------------------------------------


\section{Algorithmus}

Durch die Vorarbeit im vorangegangenem Abschnitt können wir einen effizienten Algorithmus zur Lösung des Sparse PCA Kriteriums angeben. Zuerst initialisieren wir $\mat A$ mit den ersten $k$ Hauptachsen. Anschließend minimieren wir abwechselnd über die Matrizen $\mat A$ und $\mat B$ bis ein geeignetes Konvergenzkriterium erfüllt ist oder wir eine maximale Anzahl an Iterationen erreicht haben. Durch abschließende Normalisierung der Spalten von $\mat B$ erhalten wir die dünnbesetzten Hauptachsen. Eine Übersicht haben wir in Algorithmus \ref{spca_algorithm} erstellt.

\begin{algorithm}[tbh]
    \caption{Sparse Principal Component Analysis}
    \label{spca_algorithm}
    \begin{algorithmic}[1]
        \Procedure{SPCA}{$\mat A, \mat B, k, \lambda_2, \lambda_{1,j}$}
        	\State $\mat A \gets \mat V[,1 \colon k]$, die ersten $k$ Hauptachsen
            \While{nicht konvergiert}
                \State Gegeben festes $\mat A = [\alpha_1, \ldots, \alpha_k]$, löse das elastic net Problem
                $$\widehat{\beta}_j = \argmin_{\beta_j} \norm{\mat X \alpha_j - \mat X \beta_j}^{2} + \lambda_2 \norm{\beta_j}^2 + \lambda_{1,j}\norm{\beta_j}_{1} \quad \text{für } j = 1, \ldots, k$$
                \State Gegeben festes $\mat B = [\beta_1, \ldots, \beta_k]$, berechne die Singulärwerzerlegung von $$\mat X^T \mat X \mat B = \mat U \mat D \mat V^T$$
                $$\mat A \gets \mat U \mat V^T$$
            \EndWhile
            \State $\widehat{V}_j \gets \frac{\beta_j}{\norm{\beta_j}}$ for $j = 1, \ldots, k$
        \EndProcedure
    \end{algorithmic}
\end{algorithm} 

Es stellt sich nun die Frage nach einem passendem Abbruchkriterium. Da für uns am Schluss des Algorithmus nur die dünnbesetzten Hauptachsen relevant sind, liegt es nahe ein Konvergenzkriterium für $\mat B$ zu wählen. Zou et al. brechen die Iteration in ihrer Implementierung ab, falls
$$\max_{\substack{1 \leq i \leq p \\ 1 \leq j \leq k}} \left| \frac{\beta_{ij}^{(l+1)}}{\norm{\beta_i}} - \frac{\beta_{ij}^{(l)}}{\norm{\beta_i}} \right| < \epsilon ,$$
wobei $\beta_{ij}^{(l)}$ der $j$-te Eintrag der dünnbesetzten Hauptachse $\beta_i$ in der $l$-ten Iteration ist. Sobald also die Änderung in $\mat B$ klein genug ist, kann die while-Schleife beendet werden. Um die Laufzeit des Algorithmus zu beschränken, ist es sinnvoll ein zusätzliches Abbruchkriterium zu definieren. So werden wir bei Anwendung des Algorithmus eine maximale Anzahl an Iterationen $l_{max}$ festlegen, die nicht überschritten werden darf.


%----------------------------------------------------------------------------------------
%	Komplexität
%----------------------------------------------------------------------------------------


\section{Komplexität}
\label{complexity}

Wir werden uns nun mit der Komplexität von Algorithmus \ref{spca_algorithm} beschäftigen. Dabei werden wir wieder einmal zwischen den Fällen $n > p$ und $p \gg n$ unterscheiden.

\subsubsection{Fall: $\mathbf{n > p}$}

In diesem Fall lässt sich ein Trick für die Berechnung von (\ref{sub_problem_enet}) anwenden. Indem wir
\begin{align}
\label{sub_problem_enet_covariance}
\widehat{\beta}_j & = \argmin_{\beta_j} \norm{Y_j^* - \mat X \beta_j}_2^2 + \lambda_2 \norm{\beta_j}_2^2 + \lambda_{1,j}\norm{\beta_j}_1 \nonumber\\
& = \argmin_{\beta_j} (\alpha_j - \beta_j)^T \mat X^T \mat X (\alpha_j - \beta_j) + \lambda_2 \norm{\beta_j}_2^2 + \lambda_{1,j}\norm{\beta_j}_1
\end{align}
umformen, hängen beide Subprobleme (\ref{sub_problem_enet}) und (\ref{sub_problem_procrustes}) nur von der Kovarianzmatrix $\mat X^T \mat X$ ab. Daher ist es sinnvoll, diese vorab zu berechnen, um die Anzahl an Multiplikationen je Iteration zu verringern. Allerdings ist (\ref{sub_problem_enet_covariance}) mit $\mat X^T \mat X$ nicht direkt ein Elastic Net Problem. Definieren wir $Y^{**} = (\mat{X}^T\mat X)^{\frac{1}{2}} \alpha_j$ und $\mat X^{**} = (\mat{X}^T\mat X)^{\frac{1}{2}}$ können wir (\ref{sub_problem_enet_covariance}) aber in ein Elastic Net Problem transformieren
$$\widehat{\beta}_j = \argmin_{\beta} \norm{Y^{**} - \mat X^{**} \beta}_2^2 + \lambda_2 \norm{\beta}_2^2 + \lambda_{1,j}\norm{\beta}_1.$$
Um die Laufzeit des Algorithmus zu bestimmen, wenden wir uns Tabelle \ref{complexity_calculation} zu, welche die verschiedenen Berechnungsschritte zeigt. Die Initialisierung von $\mat A$ durch die ersten $k$ Hauptachsen wird durch eine abgeschnittene Singulärwertzerlegung von $\mat X$ berechnet. Zudem berechnen wir vorab die Kovarianzmatrix $\mat X^T \mat X$. Pro Iteration lösen wir $k$ Elastic Net Probleme und ein Procrustes Rotationsproblem. Insgesamt ergibt sich aufgrund $k < \min \{n,p\}$ und $j < p$ eine Laufzeit von $np^2 + mk\mathcal{O}(p^3)$, wobei $m$ die Anzahl der Iterationen ist.

\begin{table}
\centering
\begin{tabular}{ll}
Berechnung & Komplexität\\\hline\addlinespace
Singulärwertzerlegung von $\mat X$ & $\mathcal{O}(npk)$\\
$\mat X^T\mat X$ & $\mathcal{O}(np^2)$\\
$(\mat X^T\mat X) \mat B$ & $\mathcal{O}(p^2k)$\\
$\mat X^T (\mat X \mat B)$ & $\mathcal{O}(npk)$\\
Singulärwertzerlegung von $\mat X^T\mat X \mat B$ & $\mathcal{O}(pk^2)$\\
Elastic Net Problem & $\mathcal{O}(pnj + j^3)$
\end{tabular}
\caption{Die Tabelle zeigt die Komplexität für die in Algorithmus \ref{spca_algorithm} vorkommenden Berechnungen. Dabei ist $j$ die Anzahl von Null verschiedener Koeffizienten.}
\label{complexity_calculation}
\end{table}

\subsubsection{Fall: $\mathbf{p \gg n}$}

Für diesen Fall ist eine Berechnung der Kovarianzmatrix $\mat X^T\mat X$ nicht mehr sinnvoll, da dies eine $p \times p$-Matrix ist. Daher werden wir $\mat X^T(\mat X \mat B)$ in jeder Iteration naiv berechnen. Die Laufzeit des Algorithmus wird in diesem Fall von der Lösung der $k$ Elastic Nets dominiert, besonders wenn wir viele von Null verschiedene Einträge zulassen. Insgesamt ergibt sich die Komplexität $mk\mathcal{O}(pnj + j^3)$. Für große $j$ bzw. $p$ können die Berechnungskosten somit sehr hoch sein, weshalb wir im folgenden Abschnitt einen speziellen Sparse PCA Algorithmus kennenlernen werden.


%----------------------------------------------------------------------------------------
%	Numerische Lösung im Fall p >> n
%----------------------------------------------------------------------------------------

\section{Numerische Lösung im Fall $\mathbf{p \gg n}$}
\label{numerical_solution_p_greater_n}

Für viele Anwendungen kann die Anzahl an Variablen die Anzahl an Beobachtungen deutlich übersteigen. Um auch in diesem Fall eine effiziente Berechnung zu ermöglichen, formulieren wir einen Spezialfall des Algorithmus \ref{spca_algorithm}.

Dazu beobachten wir, dass Theorem \ref{pca_regression_formulation_ridge} für alle $\lambda_2 > 0$ gilt. Für $\lambda_2 \to \infty$ erhalten wir eine interessante Charakterisierung.
\begin{thm} \label{spca_p_greater_n}
Seien $\frac{\widehat{\beta}_j(\lambda_2)}{\norm{\widehat{\beta}_j(\lambda_2)}}$ die dünnbesetzten Hauptachsen aus (\ref{spca_criterion})
und $(\tilde{\mat A}, \tilde{\mat B})$ die Lösung des Optimierungsproblems
\begin{align}
\label{spca_p_greater_n_problem}
\begin{split}
(\tilde{\mat A}, \tilde{\mat B}) = \argmin_{\mat A, \mat B} -2\spur{\mat A^T\mat X^T\mat X \mat B} + \sum_{j=1}^k \norm{\beta_j}_2^2 + \sum_{j=1}^k \lambda_{1,j}\norm{\beta_j}_1\\
\text{unter der Nebenbedingung, dass } \mat A^T \mat A = \mat I_{k \times k}.
\end{split}
\end{align}
Für $\lambda_2 \to \infty$ konvergieren die dünnbesetzten Hauptachsen $\frac{\widehat{\beta}_j(\lambda_2)}{\norm{\widehat{\beta}_j(\lambda_2)}} \to \frac{\tilde{\beta}_j}{\norm{\tilde{\beta}_j}}.$
\end{thm}

Daher können wir das vereinfachte Optimierungsproblem (\ref{spca_p_greater_n_problem}) benutzen, um die Elastic Net Probleme für den Spezialfall $\lambda_2 = \infty$ zu lösen. Fixieren wir wie in Algorithmus \ref{spca_algorithm} die Matrix $\mat A$, verbleiben wir mit dem Problem
\begin{align}
\label{spca_p_greater_n_problem_fixed_A}
\widehat{\beta}_j = \argmin_{\beta_j} -2 \alpha_j^T(\mat X^T\mat X)\beta_j + \norm{\beta_j}_{2}^{2} + \lambda_{1,j}\norm{\beta_j}_1.
\end{align}
Für (\ref{spca_p_greater_n_problem_fixed_A}) können wir aufgrund des Wegfalls von $\lambda_2$ eine explizite Lösung angeben
\begin{align}
\label{spca_p_greater_n_solution}
\widehat{\beta}_j = \operatorname{soft}_{\frac{\lambda_{1,j}}{2}}(\alpha_j^T\mat X^T\mat X) = \left(|\alpha_j^T\mat X^T\mat X| - \frac{\lambda_{1,j}}{2}\right)_+ \operatorname{Sign}(\alpha_j^T\mat X^T\mat X).
\end{align}
Daher ersetzen wir die Berechnung in Schritt 4 von Algorithmus \ref{spca_algorithm} durch (\ref{spca_p_greater_n_solution}) falls $p \gg n$. Für den hochdimensionalen Fall kann somit ein effizientes Verfahren gewährleistet werden.

UST totally ignores the dependence between predictors and treats them as independent variables. Although this may be considered illegitimate, UST and its variants are used in other methods such as significance analysis of microarrays (Tusher et al., 2001) and the nearest shrunken
centroids classifier (Tibshirani et al., 2002), and have shown good empirical performance


%----------------------------------------------------------------------------------------
%	Eigene Implementierung in Python
%----------------------------------------------------------------------------------------


\section{Eigene Implementierung in Python}

Momentan existieren Implementierungen der dünnbesetzten Hauptkomponentenanalyse in R und Python. Zou et al. stellen das elasticnet package mit einer spca-Funktion, welche auf ihrem Ansatz beruht, in der Programmiersprache R zur Verfügung. Für die Lösung des Subproblems (\ref{sub_problem_enet}) wird der LARS-EN Algorithmus gewählt, welche eine Erweiterung des LARS-Algorithmus für das Elastic Net ist \cite{zou_elasticnet}. Dagegen bietet scikit-learn eine Sparse PCA-Variante in Python, welche auf \cite{jenatton} zurückgeht und einen anderen Ansatz verfolgt.

Um ein genaues Verständnis der Ergebnisse zu garantieren, haben wir uns dazu entschieden, eine eigene Implementierung in Python vorzunehmen, die auf dem Ansatz von Zou et al. beruht. Es wurde kritisch überprüft, dass der von uns implementierte Code korrekte Ergebnisse erzielt. Dazu haben  wir den Pitprops Datensatz aus \cite{zou_sparsepca}, welcher oft als Benchmark genutzt wird, und zusätzlich den eigenen Datensatz, welchen wir in Kapitel \ref{application} beschreiben, verwendet. Gegenüber der Implementierung im elasticnet package haben wir zwei entscheidende Änderungen vorgenommen, welche die Laufzeit in der Praxis verkürzen. Statt das Subproblem (\ref{sub_problem_enet}) mit LARS-EN zu lösen, wählen wir ein Koordinaten-Abstiegsverfahren, welches wir in Abschnitt \ref{elastic_net} beschrieben haben. Des Weiteren wird in der Implementierung von Zou et al. die Gram-Matrix $\mat X^T \mat X$ immer vorab berechnet, um für das Subproblem (\ref{sub_problem_procrustes}) nur eine Multiplikation pro Iteration $(\mat X^T \mat X) \mat B$ durchführen zu müssen. Da die Gram-Matrix aber in $\mathbb{R}^{p \times p}$ liegt, ist es für den Fall $p \gg n$ sinnvoller, $\mat X^T (\mat X \mat B)$ in jeder Iteration naiv zu berechnen, damit keine $p \times p$-Matrix zwischengespeichert werden muss. Dies ermöglicht für unseren hochdimensionalen Datensatz eine Laufzeit, die etwa um den Faktor $4$ besser ist.

Bezüglich der Aufrufstruktur der Methode haben wir eine Reparametrisierung vorgenommen, um die Notation mit der ElasticNet-Klasse in scikit-learn zu vereinheitlichen. Ähnlich wie in Abschnitt \ref{comparison_linear_models} definieren wir
\begin{align}
\lambda = \frac{2\lambda_2 + \lambda_1}{2n} \quad \text{und} \quad \alpha = \frac{\lambda_1}{2\lambda_2 + \lambda_1}
\end{align}
wobei $\lambda$ die Stärke der Bestrafung und $\alpha$ das Verhältnis des $\ell_1$ zum $\ell_2$-Strafterm beschreibt.