% Main chapter title
\chapter{Einführung}

\cite{elad}
\cite{foucart}
\cite{hastie_elements}
\cite{gribonval}
\cite{jenatton}
\cite{johnstone}
\cite{yata}
\cite{mairal}
\cite{tibshirani_lasso}
\cite{tibshirani_uniqueness}
\cite{zou_elasticnet}
\cite{zou_sparsepca}
\cite{zou_overview}
\cite{efron_lars}

% Chapter label
\label{introduction}

\section{Motivation}

\section{Dimensionsreduktionsverfahren}

High dimensionality means that the dataset has a large number of features. The primary problem associated with high-dimensionality in the machine learning field is model overfitting, which reduces the ability to generalize beyond the examples in the training set. Richard Bellman described this phenomenon in 1961 as the Curse of Dimensionality where “Many algorithms that work fine in low dimensions become intractable when the input is high-dimensional. “

Let’s say that you want to predict what the gross domestic product (GDP) of the United States will be for 2017. You have lots of information available: the U.S. GDP for the first quarter of 2017, the U.S. GDP for the entirety of 2016, 2015, and so on. You have any publicly-available economic indicator, like the unemployment rate, inflation rate, and so on. You have U.S. Census data from 2010 estimating how many Americans work in each industry and American Community Survey data updating those estimates in between each census. You know how many members of the House and Senate belong to each political party. You could gather stock price data, the number of IPOs occurring in a year, and how many CEOs seem to be mounting a bid for public office. Despite being an overwhelming number of variables to consider, this just scratches the surface.
TL;DR — you have a lot of variables to consider.
If you’ve worked with a lot of variables before, you know this can present problems. Do you understand the relationships between each variable? Do you have so many variables that you are in danger of overfitting your model to your data or that you might be violating assumptions of whichever modeling tactic you’re using?
You might ask the question, “How do I take all of the variables I’ve collected and focus on only a few of them?” In technical terms, you want to “reduce the dimension of your feature space.” By reducing the dimension of your feature space, you have fewer relationships between variables to consider and you are less likely to overfit your model. (Note: This doesn’t immediately mean that overfitting, etc. are no longer concerns — but we’re moving in the right direction!)
Somewhat unsurprisingly, reducing the dimension of the feature space is called “dimensionality reduction.” There are many ways to achieve dimensionality reduction, but most of these techniques fall into one of two classes:
Feature Elimination
Feature Extraction

2. Why is Dimensionality Reduction required?
Here are some of the benefits of applying dimensionality reduction to a dataset:

Space required to store the data is reduced as the number of dimensions comes down
Less dimensions lead to less computation/training time
Some algorithms do not perform well when we have a large dimensions. So reducing these dimensions needs to happen for the algorithm to be useful
It takes care of multicollinearity by removing redundant features. For example, you have two variables – ‘time spent on treadmill in minutes’ and ‘calories burnt’. These variables are highly correlated as the more time you spend running on a treadmill, the more calories you will burn. Hence, there is no point in storing both as just one of them does what you require
It helps in visualizing data. As discussed earlier, it is very difficult to visualize data in higher dimensions so reducing our space to 2D or 3D may allow us to plot and observe patterns more clearly

\section{Sparse Approximations / Representations}

\section{Interpretierbarkeit}

\section{Compressed Sensing Beispiel}